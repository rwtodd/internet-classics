\documentclass[12pt,letterpaper]{article}
\renewcommand{\rmdefault}{ptm}

\title{Why Adventure Games Suck And What We Can Do About It}
\date{1989}
\author{Ron Gilbert}

\newcommand{\game}[1]{\textsl{#1}}

\begin{document}

\maketitle

\noindent Of all the different types of games, the ones I most enjoy playing are
adventure/story games. It is no surprise that this is also the genre for which
I design. I enjoy games in which the pace is slow and the reward is for
thinking and figuring, rather than quick reflexes. The element that brings
adventure games to life for me is the stories around which they are woven. When
done right, it is a form of storytelling that can be engrossing in a way that
only interaction can bring. The key here is ``done right,'' which it seldom is.

One of my pet peeves is the recent trend to call story games ``Interactive
Movies.'' They are interactive, but they are not movies. The fact that people
want to call them movies just points out how lost we are. What we need to do is
to establish a genre for our works that we can call our own. Movies came from
stage plays, but the references are long lost and movies have come into their
own. The same thing needs to happen to story games.

The desire to call them Interactive Movies comes from a couple of places. The
first is Marketing. It is the goal of narrow-minded marketing to place
everything into a category so it will be recognizable. These people feel that
the closest things to story games are movies. The other source for the name
Interactive Movie is what I call ``Hollywood Envy.'' A great number of people in
this business secretly (and not so secretly) wish they were making movies, not
writing video games. Knock it off! If you really want to make movies, then go
to film school and leave the game designing to people who want to make games.

Story games are not movies, but the two forms do share a great deal. It is not
fair to completely ignore movies. We can learn a lot from them about telling
stories in a visual medium. However, it is important to realize that there are
many more differences than similarities. We have to choose what to borrow and
what to discover for ourselves.

The single biggest difference is interaction. You can't interact with a movie.
You just sit in the theater and watch it. In a story game, the player is given
the freedom to explore the story. But the player doesn't always do what the
designer intended, and this causes problems. It is hard to create a cohesive
plot when you have no idea what part of the story the player will trip over
next. This problem calls for a special kind of storytelling, and we have just
begun to scratch the surface of this art form.

There is a state of mind called ``suspension of disbelief.'' When you are
watching a movie, or reading a good book, your mind falls into this state. It
occurs when you are pulled so completely into the story that you no longer
realize you are in a movie theater or sitting at your couch, reading. When the
story starts to drag, or the plots begins to fall apart, the suspension of
disbelief is lost. You soon start looking around the theater, noticing the
people in front of you or the green exit sign. One way I judge a movie is by
the number of times I realized I was in a theater.

The same is true of story games (as well as almost all other kinds of games).
As the story builds, we are pulled into the game and leave the real world
behind. As designers, our job is to keep people in this state for as long as
possible. Every time the player has to restore a saved game, or pound his head
on the desk in frustration, the suspension of disbelief is gone. At this time
he is most likely to shut off the computer and go watch TV, at which point we
all have lost.

I have created a set of rules of thumb that will minimize the loss of
suspension of disbelief. As with any set of rules, there are always exceptions.
In my designs, I hope that if these rules cannot be followed, it is for
artistic reasons and not because I am too lazy to do it right. In \game{Maniac
Mansion}, in one place or another, I violated all but one of these rules. Some
of them were violated by design, others by sloppiness. If I could redesign
\game{Maniac Mansion}, all the violations would be removed and I'd have a much better
game.

Some people say that following these rules makes the games too easy to play. I
disagree. What makes most games tough to play is that the puzzles are arbitrary
and unconnected. Most are solved by chance or repetitive sessions of typing
``\emph{light candle with match},'' ``\emph{light paper with match},'' ``\emph{light rug with match},''
until something happens. This is not tough game play, this is masturbation. I
played one game that required the player to drop a bubble gum wrapper in a room
in order to get a trap door to open (object names have been changed to protect
the guilty). What is the reasoning? There is none. It's an advanced puzzle, I
was told.

Here, then, are Gilbert's Rules of Thumb:

\section*{End Objective Needs to be Clear} It's OK if the objective changes in
mid-game, but at the beginning the player should have a clear vision as to what
he or she is trying to accomplish. Nothing is more frustrating than wandering
around wondering what you should be doing and if what you have been doing is
going to get you anywhere. Situations where not knowing what's going on can be
fun and an integral part of the game, but this is rare and difficult to pull
off.

\section*{Sub-goals Need to be Obvious} Most good adventure games are broken up
into many sub-goals. Letting the player know at least the first sub-goal is
essential in hooking them. If the main goal is to rescue the prince, and the
player is trapped on an island at the beginning of the game, have another
character in the story tell them the first step: get off the island. This is
just good storytelling. Ben Kenobi pretty much laid out Luke's whole journey in
the first twenty minutes of \game{Star Wars}. This provided a way for the audience to
follow the progress of the main character. For someone not used to the
repetitive head-banging of adventure games, this simple clue can mean the
difference between finishing the game and giving up after the first hour. It's
very easy when designing to become blind to what the player doesn't know about
your story.

\section*{Live and learn} As a rule, adventure games should be able to be
played from beginning to end without ``dying'' or saving the game if the player
is very careful and very observant. It is bad design to put puzzles and
situations into a game that require a player to die in order to learn what not
to do next time. This is not to say that all death situations should be
designed out. Danger is inherent in drama, but danger should be survivable if
the player is clever.

As an exercise, take one complete path through a story game and then tell it to
someone else, as if it were a standard story. If you find places where the main
character could not have known a piece of information that was used (the
character who learned it died in a previous game), then there is a hole in the
plot.

\section*{Backwards Puzzles} The backwards puzzle is probably the one thing
that bugs me more than anything else about adventure games. I have created my
share of them; and as with most design flaws, it's easier to leave them in than
to redesign them. The backwards puzzle occurs when the solution is found before
the problem. Ideally, the crevice should be found before the rope that allows
the player to descend. What this does in the player's mind is set up a
challenge. He knows he need to get down the crevice, but there is no route. Now
the player has a task in mind as he continues to search. When a rope is
spotted, a light goes on in his head and the puzzle falls into place. For a
player, when the design works, there is nothing like that experience.

\section*{I Forgot to Pick It Up} This is really part of the backwards puzzle
rule, but in the worst way. Never require a player to pick up an item that is
used later in the game if she can't go back and get it when it is needed. It is
very frustrating to learn that a seemingly insignificant object is needed, and
the only way to get it is to start over or go back to a saved game. From the
player's point of view, there was no reason for picking it up in the first
place. Some designers have actually defended this practice by saying that,
``adventure games players know to pick up everything.'' This is a cop-out. If the
jar of water needs to be used on the spaceship and it can only be found on the
planet, create a use for it on the planet that guarantees it will be picked up.
If the time between the two uses is long enough, you can be almost guaranteed
that the player forgot she even had the object.

The other way around this problem is to give the player hints about what she
might need to pick up. If the aliens on the planet suggest that the player find
water before returning to the ship, and the player ignores this advice, then
failure is her own fault.

\section*{Puzzles Should Advance the Story} There is nothing more frustrating
than solving pointless puzzle after pointless puzzle. Each puzzle solved should
bring the player closer to understanding the story and game. It should be
somewhat clear how solving this puzzle brings the player closer to the
immediate goal. What a waste of time and energy for the designer and player if
all the puzzle does is slow the progress of the game.

\section*{Real Time is Bad Drama} One of the most important keys to drama is
timing. Anyone who has designed a story game knows that the player rarely does
anything at the right time or in the right order. If we let the game run on a
clock that is independent from the player's actions, we are going to be
guaranteed that few things will happen with dramatic timing. When Indiana~Jones
rolled under the closing stone door and grabbed his hat just in time, it sent a
chill and a cheer through everyone in the audience. If that scene had been done
in a standard adventure game, the player would have been killed the first four
times he tried to make it under the door. The next six times the player would
have been too late to grab the hat. Is this good drama? Not likely. The key is
to use Hollywood time, not real time. Give the player some slack when doing
time-based puzzles. Try to watch for intent. If the player is working towards
the solution and almost ready to complete it, wait. Wait until the hat is
grabbed, then slam the door down. The player thinks they ``just made it'' and
consequently a much greater number of players get the rush and excitement. When
designing time puzzles I like to divide the time into three categories. 10\% of
the players will do the puzzle so fast and efficiently that they will finish
with time to spare. Another 10\% will take too much time and fail, which leaves
80\% of the people to brush through in the nick of time.

\section*{Incremental Reward} The player needs to know that she is achieving.
The fastest way to turn a player off is to let the game drag on with no
advancement. This is especially true for people who are playing adventure games
for the first time. In graphics adventures the reward often comes in the form
of seeing new areas of the game. New graphics and characters are often all that
is needed to keep people playing. Of course, if we are trying to tell a story,
then revealing new plot elements and twists can be of equal or greater value.

\section*{Arbitrary Puzzles} Puzzles and their solutions need to make sense.
They don't have to be obvious, just make sense. The best reaction after solving
a tough puzzle should be, ``Of course, why didn't I think of that sooner!'' The
worst, and most often heard after being told the solution is, ``I never would
have gotten that!'' If the solution can only be reached by trial and error or
plain luck, it's a bad puzzle.

\section*{Reward Intent} The object of these games is to have fun. Figure out
what the player is trying to do. If it is what the game wants, then help the
player along and let it happen. The most common place this fails is in playing
a meta-game called ``second-guess the parser.'' If there is an object on the
screen that looks like a box, but the parser is waiting for it to be called a
mailbox, the player is going to spend a lot of time trying to get the game to
do a task that should be transparent. In parser-driven games, the key is to
have lots of synonyms for objects. If the game is a graphics adventure, check
proximity of the player's character. If the player is standing right next to
something, chances are they are trying to manipulate it. If you give the player
the benefit of the doubt, the game will be right more than wrong. On one
occasion, I don't know how much time I spent trying to tie a string on the end
of a stick. I finally gave up, not knowing if I was wording the sentence wrong
or if it was not part of the design. As it turned out, I was wording it wrong.

\section*{Unconnected Events} In order to pace events, some games lock out
sections until certain events have happened. There is nothing wrong with this,
it is almost a necessity. The problem comes when the event that opens the new
section of the world is unconnected. If the designer wants to make sure that
six objects have been picked up before opening a secret door, make sure that
there is a reason why those six objects would affect the door. If a player has
only picked up five of the objects and is waiting for the door to open (or
worse yet, trying to find a way to open the door), the act of getting the
flashlight is not going to make any sense in relation to the door opening.

\section*{Give the Player Options} A lot of story games employ a technique that
can best be described as caging the player. This occurs when the player is
required to solve a small set of puzzles in order to advance to the next
section of the game, at which point she is presented with another small set of
puzzles. Once these puzzles are solved, in a seemingly endless series of cages,
the player enters the next section. This can be particularly frustrating if the
player is unable to solve a particular puzzle. The areas to explore tend to be
small, so the only activity is walking around trying to find the one solution
out.

Try to imagine this type of puzzle as a cage the player is caught in, and the
only way out is to find the key. Once the key is found, the player finds
herself in another cage. A better way to approach designing this is to think of
the player as outside the cages, and the puzzles as locked up within. In this
model, the player has a lot more options about what to do next. She can select
from a wide variety of cages to open. If the solution to one puzzle stumps her,
she can go on to another, thus increasing the amount of useful activity going
on.

Of course, you will want some puzzles that lock out areas of the game, but the
areas should be fairly large and interesting unto themselves. A good indicator
of the cage syndrome is how linear the game is. If the plot follows a very
strict line, chances are the designer is caging the player along the path. It's
not easy to uncage a game, it requires some careful attention to the plot as
seen from players coming at the story from different directions. The easiest
way is to create different interactions for a given situation depending on the
order encountered.

\section*{Conclusion} If I could change the world, there are a few things I
would do, and quite frankly none of them have anything to do with computers or
games. But since this article is about games?

The first thing I'd do is get rid of save games. If there have to be save
games, I would use them only when it was time to quit playing until the next
day. Save games should not be a part of game play. This leads to sloppy design.
As a challenge, think about how you would design a game differently if there
were no save games. If you ever have the pleasure of watching a non-gameplayer
playing an adventure game you will notice they treat save game very differently
then the experienced user. Some start using it as a defense mechanism only
after being slapped in the face by the game a few times, the rest just stop
playing.

The second thing I'd change would be the price. For between forty and fifty
dollars a game, people expect a lot of play for their money. This rarely leads
to huge, deep games, but rather time-wasting puzzles and mazes. If the designer
ever thinks the game might be too short, he throws in another puzzle or two.
These also tend to be the worst thought-out and most painful to solve. If I
could have my way, I'd design games that were meant to be played in four to
five hours. The games would be of the same scope that I currently design, I'd
just remove the silly time-wasting puzzles and take the player for an intense
ride. The experience they would leave with would be much more entertaining and
a lot less frustrating. The games would still be challenging, but not at the
expense of the players patience.

If any type of game is going to bridge the gap between games and storytelling,
it is most likely going to be adventure games. They will become less puzzle
solving and more story telling, it is the blueprint the future will be made
from. The thing we cannot forget is that we are here to entertain, and for most
people, entertainment does not consist of nights and weekends filled with
frustration. The average American spends most of the day failing at the office,
the last thing he wants to do is come home and fail while trying to relax and
be entertained.

\end{document}
