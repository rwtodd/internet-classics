% palatino set
\font\twelverm=pplr at 12pt
\font\twelvesl=pplro at 12pt
\font\twelvett=cmtt12
%\font\twelvebf=pplb at 12pt
\font\biggertxt=pplr at 16pt
\font\biggerbold=pplb at 14pt

% adjust the page numbers and typical font commands
\footline={\hss\rm\folio\hss}
\def\rm{\twelverm}
\def\sl{\twelvesl}
\def\tt{\twelvett}
%\def\bf{\twelvebf}

\def\title#1{\centerline{\biggertxt #1}\par}
\def\subtitle#1{\medskip\centerline{\sl #1}}
\def\startlist{\smallskip\begingroup\narrower\parskip=6pt plus 1pt minus 1pt}%
\def\endlist{\smallskip\endgroup\noindent\ignorespaces}%
\catcode`\|=13 \def|{\begingroup\tt\def|{\endgroup}}%


\parskip=8pt plus 1pt minus 1pt
\baselineskip=14pt
\rm

\leavevmode\vskip 0.25in
\title{Message to comp.lang.fortran}
\subtitle{Glen Reesor, 1995--12--01}
\vskip 0.5in
\noindent In my (relatively) short time modifying and working with
legacy Fortran codes, I have come across a number of, shall we say,
interesting tendencies.  I attribute these tendencies to a number of
things:
\startlist%
\item{1.}%
  Historical limitations of compilers and hardware (very
  understandable---at least for dusty-deck code, but depressing
  for newly written code).
\item{2.}%
  The author being someone with zero training in how to write 
  maintainable, readable, quality software.  (In my opinion, a very
  significant factor in new Fortran software being written.)
\item{}%
  Although, I keep telling myself that item 2 is very common, I've
  lost count of the number of times I've been trying to decipher
  a code segment and decided there should be an item 3:
\item{3.}%
  The author of the code is from a different planet where their logic
  is a complete reversal (yet twisted in some way) to ours :-)
\endlist
So, in order to soothe my nerves, I've created a guide for writing
modern Fortran.  There are two especially funny things about this list
(in my opinion, of course):
\startlist%
\item{--}%
  I have encountered every single one of these (within the last 2 years)
\item{--}%
  Most of these practices are being continued at this very moment by
  some unnamed people I have encountered
\endlist
So, without further delay, I give you...
\bigskip\title{Fortran Coding Style for the Modern Programmer}
\startlist%
\item{1.} When picking variable names, pick something meaningful then remove all 
   the vowels.  If the result is longer than 6 characters, truncate as 
   required.
\item{2.} When making code changes, don't delete lines---just comment them out.
   (You might need them later.)
\item{3.} |WRITE ALL CODE IN UPPERCASE IT LOOKS BETTER THAT WAY|.
\item{4.} There is no need to use comments because,
``Fortran is self-documenting.'' (Don't forget rule~1.)
\item{5.} Another good reason to leave out comments is that they cause slower
   execution of your program.
\item{6.} If you must include comments, they are easiest to read if you alternate
   comment lines and code lines (with the same level of indentation), with 
   no blank lines or `dash' lines in between.
\item{7.} When deciding what the condition should be in an
``|IF|--|THEN|--|ELSE|'' statement, decide what is most logical for a
human to understand, then reverse it.
\item{8.} When deciding whether to put variables in common blocks or not, choose
   one of the following:%
      \startlist%
      \item{a)} Pass all variables on argument lists to subroutines, and put none
         in common blocks.  That way you know exactly where every variable
         came from.  It has the added benefit of reminding you to add 
         variables required in one routine to all routines called before it.
      \item{b)} Put all variables in common blocks, and put none in the
      subroutine argument list.
      \item{c)} There is no (c)---you must use either (a) or (b).
      \endlist
\item{9.} Write your code using implicit typing---that way you don't have to type
   as much.
\item{10.} To make yourself feel at home, always refer to `cards', `fields',
`decks' and `core memory.'
\item{11.} When using variables, pick one name when you calculate it, assign
    it to another one for use in an argument list to another subroutine,
    and call it something else inside the called routine.
    (This also has the added benefit of increasing your job security.)
\item{12.} When designing the format of your |ASCII| inputfile, base it on
a rigid column structure so you don't have to parse any keywords in 
the inputfile.
\item{13.} To make your program footprint smaller, use the same arrays for
different things, depending on which part of the program you're in.
\item{14.} When you realize that the same code segment is being written many
    times, it is best to cut-and-paste multiple times rather than risk a 
    mistake creating a subroutine.
\item{15.} To make code more readable, write it like this:
{\smallskip\narrower\parskip=2pt\obeylines\tt%
    if (a .eq.\ 1) then
    \quad result = result + 1
    else if (a .eq.\ 2) then
    \quad result = result + 2
    else if (a .eq.\ 3) then
    \quad result = result + 3
    else
    \quad result = result + a
    endif\smallskip}%    
\item{16.} A blank line must start with a `|C|.'
\item{17.} The Golden Rule: 
\item{} Regardless of the capacity of the machine your code is to run on,
      it is more important to make it small and run blindingly fast,
      than to maximize program readability and maintainability.
\endlist

\bye
