\documentclass[12pt,letterpaper]{article}
\title{Message to comp.lang.fortran}
\author{Glen Reesor}
\date{December 1, 1995}

\newcommand{\acro}[1]{{\small #1\spacefactor1000}}

\begin{document}
\maketitle

\noindent In my (relatively) short time modifying and working with
legacy Fortran codes, I have come across a number of, shall we say,
interesting tendencies.  I attribute these tendencies to a number of
things:

\begin{enumerate}
\item Historical limitations of compilers and hardware (very
  understandable --- at least for dusty-deck code, but depressing
  for newly written code).

\item The author being someone with zero training in how to write 
  maintainable, readable, quality software.  (In my opinion, a very
  significant factor in new Fortran software being written.)

  Although, I keep telling myself that item 2 is very common, I've
  lost count of the number of times I've been trying to decipher
  a code segment and decided there should be an item 3:

\item The author of the code is from a different planet where their logic
  is a complete reversal (yet twisted in some way) to ours :-)
\end{enumerate}
So, in order to soothe my nerves, I've created a guide for writing
modern Fortran.  There are two especially funny things about this list
(in my opinion, of course):

\begin{itemize}
\item[--] I have encountered every single one of these (within the last 2 years)
\item[--] Most of these practices are being continued at this very moment by
  some unnamed people I have encountered
\end{itemize}

\noindent So, without further delay, I give you \ldots
\bigbreak
\begin{center}
\large Fortran Coding Style for the Modern Programmer
\end{center}

\begin{enumerate}
\item When picking variable names, pick something meaningful then remove all 
   the vowels.  If the result is longer than 6 characters, truncate as 
   required.
\item When making code changes, don't delete lines --- just comment them out.
   (You might need them later.)
\item {\ttfamily WRITE ALL CODE IN UPPERCASE IT LOOKS BETTER THAT WAY}
\item There is no need to use comments:
``Fortran is self-documenting.'' (Don't forget rule~1.)
\item Another good reason to leave out comments is that they cause slower
   execution of your program.
\item If you must include comments, they are easiest to read if you alternate
   comment lines and code lines (with the same level of indentation), with 
   no blank lines or `dash' lines in between.
\item When deciding what the condition should be in an
``\texttt{IF}--\texttt{THEN}--\texttt{ELSE}'' statement, decide what is most logical for a
human to understand, then reverse it.
\item When deciding whether to put variables in common blocks or not, choose
   one of the following:%

   \begin{enumerate}
      \item Pass all variables on argument lists to subroutines, and put none
         in common blocks.  That way you know exactly where every variable
         came from.  It has the added benefit of reminding you to add 
         variables required in one routine to all routines called before it.
      \item Put all variables in common blocks, and put none in the
      subroutine argument list.
      \item There is no (c) --- you must use either (a) or (b).
      \end{enumerate}
\item Write your code using implicit typing --- that way you don't have to type
   as much.
\item To make yourself feel at home, always refer to `cards', `fields',
`decks' and `core memory.'
\item When using variables, pick one name when you calculate it, assign
    it to another one for use in an argument list to another subroutine,
    and call it something else inside the called routine.
    (This also has the added benefit of increasing your job security.)
\item When designing the format of your \acro{ASCII} inputfile, base it on
a rigid column structure so you don't have to parse any keywords in 
the inputfile.
\item To make your program footprint smaller, use the same arrays for
different things, depending on which part of the program you're in.
\item When you realize that the same code segment is being written many
    times, it is best to cut-and-paste multiple times rather than risk a 
    mistake creating a subroutine.
\item To make code more readable, write it like this:

{\ttfamily
\begin{verbatim}
  if (a .eq. 1) then
     result = result + 1
  else if (a .eq. 2) then
     result = result + 2
  else if (a .eq. 3) then
     result = result + 3
  else
     result = result + a
  endif
\end{verbatim}}

\item A blank line must start with a `\texttt{C}.'
\item The Golden Rule: 

      Regardless of the capacity of the machine your code is to run on,
      it is more important to make it small and run blindingly fast,
      than to maximize program readability and maintainability.
\end{enumerate}

\end{document}
