\documentclass[10pt]{article}
\usepackage[letterpaper,total={5.7in,8.5in}]{geometry}
\newcommand{\hexnum}[1]{\texttt{#1}}
\newcommand{\prog}[1]{\texttt{#1}}
\newcommand{\strong}[1]{\textbf{#1}}
\newcommand{\mc}{\small}
\newcommand{\acro}[1]{{\mc #1\spacefactor1000}}
\def\CEE/{\acro C}%
\def\UNIX/{{\mc U\kern-.05emNIX\spacefactor1000}}%
\def\ASCII/{\acro{ASCII}}%
\def\RSTTT/{\acro{RS-232}}%

\title{Things Every Hacker Once Knew}
\author{Eric S.~Raymond}
\date{2017}

\begin{document}
\maketitle
\tableofcontents

\pagebreak

\noindent One fine day in January 2017 I was reminded of something I had half-noticed a
few times over the previous decade. That is, younger hackers don't know the bit
structure of \ASCII/ and the meaning of the odder control characters in it.

This is knowledge every fledgling hacker used to absorb through their pores.
It's nobody's fault this changed; the obsolescence of hardware terminals and
the near-obsolescence of the \RSTTT/ protocol is what did it. Tools generate
culture; sometimes, when a tool becomes obsolete, a bit of cultural commonality
quietly evaporates. It can be difficult to notice that this has happened.

This document began as a collection of facts about \ASCII/ and related
technologies, notably hardware serial terminals and \RSTTT/ and modems. This is
lore that was at one time near-universal and is no longer. It's not likely to
be directly useful today  ---  until you trip over some piece of still-functioning
technology where it's relevant (like a \acro{GPS} puck), or it makes sense of some
old-fart war story. Even so, it's good to know anyway, for cultural-literacy
reasons.

One thing this collection has that tends to be indefinite in the minds of older
hackers is calendar dates. Those of us who lived through all this tend to have
remembered order and dependencies but not exact timing; here, I did the
research to pin a lot of that down. I've noticed that people have a tendency to
retrospectively back-date the technologies that interest them, so even if you
did live through the era it describes you might get a few surprises from
reading this.

There are lots of references to \UNIX/ in here because I am mainly attempting to
educate younger open-source hackers working on \UNIX/-derived systems such as
Linux and the \acro{BSD}s. If those terms mean nothing to you, the rest of this
document probably won't either.

\section{Hardware Context}
Nowadays, when two computers talk to each other, it's usually via \acro{TCP/IP} over
some physical layer you seldom need to care much about. And a ``terminal'' is
actually a ``terminal emulator,'' a piece of software that manages part of a
bit-mapped display and itself speaks \acro{TCP/IP}.

Before ubiquitous \acro{TCP/IP} and bit-mapped displays things were very different.
For most hackers that transition took place within a few years of 1992 -
perhaps somewhat earlier if you had access to then-expensive workstation
hardware.

Before then there were video display terminals --- \acro{VDT}s for short. In the
mid-1970s these had displaced an earlier generation of printing terminals
derived from \textbf{really} old technology called a ``teletype,'' which had evolved
around 1900 from Victorian telegraph networks. The very earliest versions of
\UNIX/ in the late 1960s were written for these printing terminals, in particular
for the Teletype Model 33 (aka \acro{ASR-33}); the ``tty'' that shows up in \UNIX/ device
names was a then-common abbreviation for ``teletype.''

(This is not the only \UNIX/ device name that is a fossil from a bygone age.
There's also \prog{/dev/lp} for the system default printer; every hacker once knew
that the ``lp'' stood for ``line printer,'' a type of line-at-a-time
electromechanical printer associated with mainframe computers of roughly the
same vintage as the \acro{ASR-33}.)

It's half-forgotten now that these \acro{VDT}s were deployed in fleets attached to
large central machines. Today a single personal computer has multiple
processors, but back then all the \acro{VDT}s and their users on a machine divided up
a single processor; this was called ``timesharing.'' Modern \UNIX/es still have
this capability, if anyone still cared to use it; the typical setup of having
many ``virtual terminals,'' accessible by pressing Ctrl-Alt-Fsomething, uses the
very mechanism that originally provided access to many terminals for
timesharing.

In those pre-Internet days computers didn't talk to each other much, and the
way teletypes and terminals talked to computers was a hardware protocol called
``\RSTTT/'' (or, if you're being pedantic, ``\acro{EIA RS-232C}'')  Before \acro{USB}, when
people spoke of a ``serial'' link, they meant \RSTTT/, and sometimes referred to
the equipment that spoke it as ``serial terminals.''

\RSTTT/ had a very long service life; it was developed in the early 1960s, not
originally for computer use but as a way for teletypewriters to talk to modems.
Though it has passed out of general use and is no longer common knowledge, it's
not quite dead even today.

I've been simplifying a bit here. There were other things besides \RSTTT/ and
serial terminals going on, notably on \acro{IBM} mainframes. But they've left many
fewer traces in current technology and its folklore. This is because the
lineage of modern \UNIX/ passes back through a now-forgotten hardware category
called a ``minicomputer,'' especially minicomputers made by the Digital Equipment
Corporation. \ASCII/, \RSTTT/ and serial terminals were part of the technology
cluster around minicomputers --- as was, for that matter, \UNIX/ itself.

Minicomputers were wiped out by workstations and workstations by descendants of
the \acro{IBM PC}, but many hackers old enough to remember the minicomputer era
(mid-1960s to mid-1980s) tend still to get a bit misty-eyed about the \acro{DEC}
hardware they cut their teeth on.

Often, however, nostalgia obscures how very underpowered those machines were.
For example: a \acro{DEC VAX 11-780} minicomputer in the mid-1980s, used for
timesharing and often supporting a dozen simultaneous users, had less than
1/1000 the processing power and less than 1/5000 times as much storage
available as a low-end smartphone does in 2017.

In fact, until well into the 1980s microcomputers ran slowly enough (and had
poor enough \acro{RF} shielding) that \textbf{this} was common knowledge: you
could put an \acro{AM} radio next to one and get a clue when it was doing
something unusual, because either fundamentals or major subharmonics of the
clock frequencies were in the 20Hz to 20KHz range of human audibility. Nothing
has run \textbf{that} slowly since the turn of the 21st century.

\section{The Strange Afterlife of the Hayes Smartmodem}
About those modems: the word is a portmanteau for ``modulator/demodulator.''
Modems allowed digital signals to pass over copper phone wires --- ridiculously
slowly by today's standards, but that's how we did our primitive wide-area
networking in pre-Internet times. It was \textbf{not} generally known back then that
modems had first been invented in the late 1950s for use in military
communications, notably the \acro{SAGE} air-defense network; we just took them for
granted.

Today modems that speak over copper or optical fiber are embedded invisibly in
the Internet access point in your basement; other varieties perform
over-the-air signal handling for smartphones and tablets. A variety every
hacker used to know about (and most of us owned) was the ``outboard'' modem, a
separate box wired to your computer and your telephone line.

Inboard modems (expansion cards for your computer) were also known (and became
wide-spread on consumer-grade computers towards the end of the modem era), but
hackers avoided them because being located inside the case made them vulnerable
to \acro{RF} noise, and the blinkenlights on an outboard were useful for diagnosing
problems. Also, most hackers learned to interpret (at least to some extent)
modem song --- the outboards made while attempting to establish a connection. The
happy song of a successful connect was identifiably different from various sad
songs of synchronization failure.

One relic of modem days is the name of the \UNIX/ \acro{SIGHUP} signal, indicating that
the controlling terminal of the user's process has disconnected. \acro{HUP} stands for
``Hang Up'' and this originally indicated a serial line drop (specifically, loss
of Data Carrier Detect) as produced by a modem hangup.

These old-fashioned modems were, by today's standards, unbelievably slow. Modem
speeds increased from 110 bits per second back at the beginning of interactive
computing to 56 kilobits per second just before the technology was effectively
wiped out by wide-area Internet around the end of the 1990s, which brought in
speeds of a megabit per second and more (20 times faster). For the longest
stable period of modem technology after 1970, about 1984 to 1991, typical speed
was 9600bps. This has left some traces; it's why surviving serial-protocol
equipment tends to default to a speed of 9600bps.

There was a line of modems called ``Hayes Smartmodems'' that could be told to
dial a number, or set parameters such as line speed, with command codes sent to
the modem over its serial link from the machine. Every hacker used to know the
``\texttt{AT}'' prefix used for commands and that, for example, \texttt{ATDT} followed by a phone
number would dial the number. Other modem manufacturers copied the Hayes
command set and variants of it became near universal after 1981.

What was \textbf{not} commonly known then is that the ``{\tt AT}'' prefix had a helpful special
property. That bit sequence (\hexnum{1+0 1000 0010 1+0 0010 1010 1+}, where the plus
suffix indicates one or more repetitions of the preceding bit) has a shape that
makes it as easy as possible for a receiver to recognize it even if the
receiver doesn't know the transmit-line speed; this, in turn, makes it possible
to automatically synchronize to that speed.

That property is still useful, and thus in 2017 the \texttt{AT} convention has survived
in some interesting places. \texttt{AT} commands have been found to perform control
functions on 3G and 4G cellular modems used in smartphones. On one widely
deployed variety, ``\texttt{AT+QLINUXCMD=}'' is a prefix that passes commands to an
instance of Linux running in firmware on the chip itself (separately from
whatever \acro{OS} might be running visibly on the phone).

\section{Preserving Core Values}
From about 1955 to 1975 --- before semiconductor memory --- the dominant technology
in computer memory used tiny magnetic doughnuts strung on copper wires. The
doughnuts were known as ``ferrite cores'' and main memory thus known as ``core
memory'' or ``core.''

\UNIX/ terminology was formed in the early 1970s, and compounds like ``in core''
and ``core dump'' survived into the semiconductor era. Until as late as around
1990 it could still be assumed that every hacker knew from where these terms
derived; even microcomputer hackers for which memory had always been
semiconductor \acro{RAM} tended to pick up this folklore rapidly on contact with \UNIX/.

After 2000, however, as multiprocessor systems became increasingly common even
on desktops, ``core'' increasingly took on a conflicting meaning as shorthand for
``processor core.'' In 2017 ``core'' can still mean either thing, but the reason
for the older usage is no longer generally understood and idioms like ``in core''
may be fading.

\section{36-bit Machines and the Persistence of Octal}
There's a power-of-two size hierarchy in memory units that we now think of as
normal --- 8 bit bytes, 16 or 32 or 64-bit words. But this did not become
effectively universal until after 1983. There was an earlier tradition of
designing computer architectures with 36-bit words.

There was a time when 36-bit machines loomed large in hacker folklore and some
of the basics about them were ubiquitous common knowledge, though cultural
memory of this era began to fade in the early 1990s. Two of the best-known
36-bitters were the \acro{DEC PDP-10} and the Symbolics 3600 Lisp machine. The
cancellation of the \acro{PDP-10} in '83 proved to be the death knell for this class
of machine, though the 3600 fought a rear-guard action for a decade afterwards.

Hexadecimal is a natural way to represent raw memory contents on machines with
the power-of-two size hierarchy. But octal (base-8) representations of machine
words were common on 36-bit machines, related to the fact that a 36-bit word
naturally divides into 12 3-bit fields naturally represented as octal. In fact,
back then we generally assumed you could tell which of the 32- or 36-bit phyla
a machine belonged in by whether you could see digits greater than 7 in a
memory dump.

Here are a few things every hacker used to know that related to these machines:
\begin{itemize}
\item 36 bits was long enough to represent positive and negative
integers to an accuracy of ten decimal digits, as was expected on mechanical
calculators of the era. Standardization on 32 bits was unsuccessfully resisted
by numerical analysts and people in scientific computing, who really missed
that last 4 bits of accuracy.

\item A ``character'' might be 9 bits on these machines, with 4 packed
to a word. Consequently, keyboards designed for them might have both a meta key
to assert bit 8 and a now-extinct extra modifier key (usually but not always
called ``Super'') that asserted bit 9. Sometimes this selected a tier of
non-\ASCII/ characters including Greek letters and mathematical symbols.

\item Alternatively, 6-bit characters might be packed 6 to a word.
There were many different 6-bit character encodings; not only did they differ
across a single manufacturer's machines, but some individual machines used
multiple incompatible encodings. This is why older non-\UNIX/ minicomputers like
the \acro{PDP-10} had a six-character limit on filenames --- this allowed an entire
filename to be packed in a single 36-bit word. If you ever see the following
acronyms it is a clue that you may have wandered into this swamp: \acro{SIXBIT},
\acro{FIELDATA}, \acro{RADIX-50}, \acro{BCDIC}.
\end{itemize}
It used also to be generally known that 36-bit architectures
explained some unfortunate features of the \CEE/~language. The original \UNIX/
machine, the \acro{PDP-7}, featured 18-bit words corresponding to half-words on larger
36-bit computers. These were more naturally represented as six octal (3-bit)
digits.

The immediate ancestor of \CEE/ was an interpreted language written on the \acro{PDP-7}
and named \acro{B}. In it, a numeric literal beginning with 0 was interpreted as
octal.

The \acro{PDP-7}'s successor, and the first workhorse \UNIX/ machine was the \acro{PDP-11}
(first shipped in 1970). It had 16-bit words --- but, due to some unusual
peculiarities of the instruction set, octal made more sense for its machine
code as well. \CEE/, first implemented on the \acro{PDP-11}, thus inherited the B octal
syntax. And extended it: when an in-string backslash has a following digit,
that was expected to lead an octal literal.

The Interdata 32, \acro{VAX}, and other later \UNIX/ platforms didn't have those
peculiarities; their opcodes expressed more naturally in hex. But \CEE/ was never
adjusted to prefer hex, and the surprising interpretation of leading 0 wasn't
removed.

Because many later languages (Java, Python, etc) copied \CEE/'s low-level lexical
rules for compatibility reasons, the relatively useless and sometimes dangerous
octal syntax besets computing platforms for which three-bit opcode fields are
wildly inappropriate, and may never be entirely eradicated.

The \acro{PDP-11} was so successful that architectures strongly influenced by it
(notably, including Intel and \acro{ARM} microprocessors) eventually took over the
world, killing off 36-bit machines.

\section{RS-232 and its Discontents}
A \acro{TCP/IP} link generally behaves like a clean stream of 8-bit bytes (formally,
octets). You get your data as fast as the network can run, and error
detection/correction is done somewhere down below the layer you can see.

\RSTTT/ was not like that. Two devices speaking it had to agree on a common line
speed --- also on how the byte framing works (the latter is why you'll see
references to ``stop bits'' in related documentation). Finally, error detection
and correction was done in-stream, sort of. \RSTTT/ devices almost always spoke
\ASCII/, and used the fact that \ASCII/ only filled 7 bits. The top bit might be,
but was not always, used as a parity bit for error detection. If not used, the
top bit could carry data.

You had to set your equipment at both ends for a specific combination of all of
these. After about 1984 anything other than ``8N1'' --- eight bits, no parity, one
stop bit --- became increasingly rare. Before that, all kinds of weird
combinations were in use. Even parity (``E'') was more common than odd (``O'') and
1 stop bit more common than 2, but you could see anything come down a wire.
And if you weren't properly set up for it, all you got was ``baud barf'' --- random
8-bit garbage rather than the character data you were expecting.

This, in particular, is one reason the \acro{API} for \prog{termios(3)}, 
the \acro{POSIX}/\UNIX/ terminal interface, has a lot of complicated options with no obvious
modern-day function. It had to be able to manipulate all these settings, and
more.

Another consequence was that passing binary data over an \RSTTT/ link wouldn't
work if parity was enabled --- the high bits would get clobbered. Other
now-forgotten wide-area network protocols reacted even worse, treating in-band
characters with \hexnum{0x80} on as control codes with results ranging from amusing to
dire. We had a term, ``8-bit clean,'' for networks and software that didn't
clobber the \hexnum{0x80}~bit. And we \textbf{needed} that term \ldots

The fragility of the \hexnum{0x80}~bit back in those old days is the now largely
forgotten reason that the \acro{MIME} encoding for email was invented (and within it
the well-known \acro{MIME64} encoding). Even the version of \acro{SMTP} current as I write
(RFC 5321) is still essentially a 7-bit protocol, though modern end points can
now optionally negotiate passing 8-bit data.

But before \acro{MIME} there was uuencode/uudecode, a pair of filters for rendering
8-bit data in 7 bits that is still occasionally used today in \UNIX/land. In this
century uuencoding has largely been replaced by \acro{MIME64}, but there are places
you can still trip over uuencoded binary archives.

Even the \RSTTT/ physical connector varied. Standard \RSTTT/ as defined in 1962
used a roughly D-shaped shell with 25 physical pins (\acro{DB-25}), way more than the
physical protocol actually required (you can support a minimal version with
just three wires, and this was actually common). Twenty years later, after the
\acro{IBM PC-AT} introduced it in 1984, most manufacturers switched to using a smaller
\acro{DB-9} connector (which is technically a \acro{DE-9} but almost nobody ever called it
that). If you look at a \acro{PC} with a serial port it is most likely to be a \acro{DB-9};
confusingly, \acro{DB-25} came to be used for printer parallel ports (which originally
had a very different connector) before those too were obsolesced by \acro{USB} and
Ethernet.

Anybody who worked with this stuff had to keep around a bunch of specialized
hardware --- gender changers, \acro{DB-25}-to-\acro{DB-9} adapters (and the reverse), breakout
boxes, null modems, and other gear I won't describe in detail because it's left
almost no traces in today's tech. Hackers of a certain age still tend to have
these things cluttering their toolboxes or gathering dust in a closet
somewhere.

The main reason to still care about any of this (other than understanding
greybeard war stories) is that some kinds of sensor and control equipment and
IoT devices still speak \RSTTT/, increasingly often wrapped inside a \acro{USB}
emulation. The most common devices that do the latter are probably \acro{GPS} sensors
designed to talk to computers (as opposed to handheld \acro{GPS}es or car-navigation
systems).

Because of devices like \acro{GPS}es, you may still occasionally need to know what an
\RSTTT/ ``handshake line'' is. These were originally used to communicate with
modems; a terminal, for example, could change the state of the \acro{DTR} (Data
Terminal Ready) line to indicate that it was ready to receive, initiate, or
continue a modem session.

Later, handshake lines were used for other equipment-specific kinds of
out-of-band signals. The most commonly re-used lines were \acro{DCD} (data carrier
detect) and \acro{RI} (Ring Indicator).

Three-wire versions of \RSTTT/ omitted these handshake lines entirely. A chronic
source of frustration was equipment at one end of your link that failed to
supply an out-of-band signal that the equipment at the other end needed. The
modern version of this is \acro{GPS}es that fail to supply their 1PPS (a
high-precision top-of-second pulse) over one of the handshake lines.

Even when your hardware generated and received handshake signals, you couldn't
necessarily trust cables to pass them. Cheap cables often failed to actually
connect all 25 (or 9) leads end-to-end. Then there were ``crossover'' or ``null
modem'' cables, which for reasons too painful to go into here crosswired the
transmit and receive lines. Of course none of these variants were reliably
labeled, so debugging an \RSTTT/ link with a random cable pulled out of a box
often involved a lot of profanity.

Another significant problem was that an \RSTTT/ device not actually sending data
was undetectable without analog-level monitoring equipment. You couldn't tell a
working but silent device from one that had come unplugged or suffered a
connection fault in its wiring. This caused no end of complications when
troubleshooting and is a major reason \acro{USB} was able to displace \RSTTT/ after
1994.

A trap for the unwary that opened up after about the year 2000 is that
peripheral connectors labeled \RSTTT/ could have one of two different sets of
voltage levels. If they're pins or sockets in a \acro{DB9} or \acro{DB25} shell, the voltage
swing between 1 and 0 bits can be as much as 50 volts, and is usually about 26.
Bare connectors on a circuit board, or chip pins, increasingly came to use
what's called ``\acro{TTL} serial'' --- same signalling with a swing of 3.3 or (less
often) 5 volts. You can't wire standard \RSTTT/ to \acro{TTL} serial directly; the link
needs a device called a ``level shifter.'' If you connect without one, components
on the \acro{TTL} side will get fried.

\RSTTT/ passed out of common knowledge in the mid- to late 1990s, but didn't
finally disappear from general-purpose computers until around 2010. Standard
\RSTTT/ is still widely used not just in the niche applications previously
mentioned, but also in point-of-sale systems diagnostic consoles on
commercial-grade routers, and debugging consoles on embedded systems. The \acro{TTL}
serial variant is often used in the latter context, especially on maker
devices.

\section{WAN Time Gone: The Forgotten Pre-Internets}
Today, the \acro{TCP/IP} Internet is very nearly the only \acro{WAN} (Wide-Area-Network) left
standing. It was not always so. From the late '70s to the mid-1990s --- but
especially between 1981 and 1991 --- there were a profusion of \acro{WAN}s of widely
varying capability. You are most likely to trip over references to these in
email archives from that time; one characteristic of it is that people
sometimes advertised multiple different network addresses in their signatures.

Every hacker over a certain age remembers either \acro{UUCP} or the \acro{BBS} scene. Many
participated in both. In those days access to the ``real'' net (\acro{ARPANET,} which
became Internet) was difficult if you weren't affiliated with one of a select
group of federal agencies, military contractors, or university research labs.
So we made do with what we had, which was modems and the telephone network.

\acro{UUCP} stands for \UNIX/ to \UNIX/ Copy Program. Between its escape from Bell Labs in
1979 and the mass-market Internet explosion of the mid-1990s, it provided slow
but very low-cost networking among \UNIX/ sites using modems and the phone
network.

\acro{UUCP} was a store-and-forward system originally intended for propagating
software updates, but its major users rapidly became email and a thing called
\acro{USENET} (launched 1981) that was the ur-ancestor of Stack Overflow and other
modern web fora. It supported topic groups for messages which, propagated from
their point of origin through \acro{UUCP}, would eventually flood to the whole
network.

In part, \acro{UUCP} and \acro{USENET} were a hack around the two-tier rate structure that
then existed for phone calls, with ``local'' being flat-rate monthly and
``long-distance'' being expensively metered by the minute. \acro{UUCP} traffic could be
relayed across long distances by local hops.

A direct descendant of \acro{USENET} still exists, as Google Groups, but was much
more central to the hacker culture before cheap Internet. Open source as we now
know it germinated in \acro{USENET} groups dedicated to sharing source code. Several
conventions still in use today, like having project metadata files named \acro{README}
and \acro{NEWS} and \acro{INSTALL}, became established there in the early 1980s --- though at
least \acro{README} was older, having been seen in the wild back on the \acro{PDP-10}.

Two key dates in \acro{USENET} history were universally known. One was the Great
Renaming in 1987, when the name hierarchy of \acro{USENET} topic groups was
reorganized. The other was the ``September that never ended'' in 1993, when the
\acro{AOL} commercial timesharing services gave its users access to \acro{USENET}. The
resulting vast flood of newbies proved difficult to acculturate.

\acro{UUCP} explains a quirk you may run across in old mailing-list archives: the
bang-path address. \acro{UUCP} links were point-to-point and you had to actually
specify the route of your mail through the \acro{UUCP} network; this led to people
publishing addresses of the form ``\ldots\texttt{!bigsite\allowbreak !foovax%
\allowbreak !barbox\allowbreak !user},'' presuming that
people who wanted to reach them would know how to reach bigsite. As Internet
access became more common in the early 1990s, addresses of the form
{\tt user@hostname} displaced bang paths. During this transition period there were
some odd hybrid mail addresses that used a ``\texttt{\%}'' to weld bang-path routing to
Internet routing.

\acro{UUCP} was notoriously difficult to configure, enough so that people who knew how
often put that skill on their CVs in the justified expectation that it could
land them a job.

Meanwhile, in the microcomputer world, a different kind of store-and-forward
evolved --- the \acro{BBS} (Bulletin-Board System). This was software running on a
computer (after 1991 usually an \acro{MS-DOS} machine) with one (or, rarely, more)
attached modems that could accept incoming phone calls. Users (typically, just
one user at a time!) would access the \acro{BBS} using a their own modem and a
terminal program; the \acro{BBS} software would allow them to leave messages for each
other, upload and download files, and sometimes play games.

The first \acro{BBS}, patterned after the community notice-board in a supermarket, was
fielded in Chicago in 1978. Over the next eighteen years over a hundred
thousand \acro{BBS}es flashed in and out of existence, typically run out of the
sysop's bedroom or garage with a spare computer.

From 1984 the \acro{BBS} culture evolved a primitive form of internetworking called
``FidoNet'' that supported cross-site email and a forum system broadly resembling
\acro{USENET}. There were also a few ports of \acro{UUCP} to \acro{DOS} personal computers, but none
gained any real traction. Thus the \acro{UUCP} and \acro{BBS} cultures remained separate
until both were ploughed under by the Internet.

During a very brief period after 1990, just before mass-market Internet,
software with \acro{BBS}-like capabilities but supporting multiple simultaneous modem
users (and often offering \acro{USENET} access) got written for low-cost \UNIX/ systems.
The end-stage \acro{BBS}es, when they survived, moved to the Web and dropped modem
access. The history of cellar.org chronicles this period.

A handful of \acro{BBS}es are still run by nostalgicists, and some artifacts from the
culture are still preserved. But, like the \acro{UUCP} network, the \acro{BBS} culture as a
whole collapsed when inexpensive Internet became widely available.

Almost the only cultural memory of \acro{BBS}es left is around a family of
file-transfer protocols --- \acro{XMODEM}, \acro{YMODEM}, and \acro{ZMODEM} --- developed shortly before
\acro{BBS}es and primarily used on them. For hackers of that day who did not cut their
teeth on minicomputers with native \acro{TCP/IP}, these were a first introduction to
concepts like packetization, error detection, and retransmission. To this day
(2018), hardware from at least one commercial router vendor (Cisco) accepts
software patches by \acro{XMODEM} upload through a serial port.

Also roughly contemporaneous with \acro{USENET} and the \acro{BBS} culture, and also
destroyed or absorbed by cheap Internet, were some commercial timesharing
services supporting dialup access by modem, of which the best known were \acro{AOL}
(America Online) CompuServe, and GEnie; others included The Source and Prodigy.
These provided \acro{BBS}-like facilities. Every hacker knew of these, though few used
them. They have left no traces at all in today's hacker culture.

One last tier of pre-Internets, operating from about 1981 to about 1991 with
isolated survivals into the 2000s, was various academic wide-area networks
using leased-line telephone links: \acro{CSNET}, \acro{BITNET}, \acro{EARN}, \acro{VIDYANET}, and others.
These generally supported email and file-transfer services that would be
recognizable to Internet users, though with different addressing schemes and
some odd quirks (such as not being 8-bit clean). They have left some traces
today. Notably, the term ``listserv'' for an electronic mailing list, still
occasionally used today, derives from an email reflector used on \acro{BITNET}.

\section{FTP, Gopher, and the Forgotten Pre-Web}
The World Wide Web went from nowhere to ubiquity during a few short years in
the early 1990s. Before that, from 1971 onwards, file transfer between Internet
sites was normally done with a tool named ftp, for the File Transfer Protocol
it used. Every hacker once knew how to use this tool.

Eventually ftp was mostly subsumed by web browsers speaking the \acro{FTP} protocol
themselves. This is why you may occasionally still see \acro{URL}s with an ``ftp:''
service prefix; this informs the browser that it should expect to speak to an
\acro{FTP} server rather than an \acro{HTTP/HTTPS} server.

There was another. The same year (1991) that Tim Berners-Lee was inventing the
World Wide Web, a group of hackers at the University of Minnesota devised
``Gopher'', a hypertext protocol that was menu-centric rather than link-centric.
For a few years Gopher competed vigorously with the early Web, and many hackers
used both. Adoption was stalled when the University decided to charge a license
fee for its implementation in early 1993. Then, early Web browsers added the
ability to speak Gopher protocol and display Gopher documents. By 2000 Gopher
was effectively dead, although a few Gopher servers are still operated in a
spirit of nostalgia and irony.

\section{Terminal confusion}
The software terminal emulators on modern \UNIX/ systems are the near-end --- and
probably final --- manifestations of a long and rather confused history. It began
with early displays sometimes called ``glass \acro{TTY}s'' because they emulated
teletypes --- but less expensively, because they didn't require consumables like
paper. The phrase ``dumb terminal'' is equivalent. The first of these was shipped
in 1969. The best-remembered of them is probably still the \acro{ADM-3} from 1975.

The very earliest \acro{VDT}s, like the \acro{ASR-33} before them, could form only upper-case
letters. An interesting hangover from these devices was that, even though most
\acro{VDT}s made after 1975 could form lower-case letters, \UNIX/ (and Linux as late as
2018) responded to a login beginning with an upper-case letter by switching to
a mode which upcased all input. If you create an account with this sort of
login name and a mixed-case password, hilarity ensues. If the password is
upper-case the hilarity is less desperate but still confusing for the user.

The classic ``smart terminal'' \acro{VDT} designs that have left a mark on later
computing appeared during a relatively short period beginning in 1975. Devices
like the Lear-Siegler \acro{ADM-3A} (1976) and the \acro{DEC VT-100} (1978) inherited the
80-character line width of punched cards (longer than the 72-character line
length of teletypes) and supported as many lines as could fit on an
approximately 4:3 screen (and in 2K bytes of display memory); they are the
reason your software terminal emulator has a 24$\times$80 or 25$\times$80 default size.

These terminals were called ``smart'' because they could interpret control codes
to do things like addressing the cursor to any point on the screen in order to
produce truly 2-dimensional displays. The ability to do bold, underline or
reverse-video highlighting also rapidly became common. Colored text and
backgrounds, however, only became available a few years before \acro{VDT}s were
obsolesced; before that displays were monochromatic. Some had crude,
low-resolution dot graphics; a few types supported black-and-white vector
graphics.

Early \acro{VDT}s used a crazy variety of control codes. One of the principal relics
of this era is the \UNIX/ terminfo database, which tracked these codes so
terminal-using applications could do abstracted operations like ``move the
cursor'' without being restricted to working with just one terminal type. The
\prog{curses(3)} library still used with software terminal emulators was originally
intended to make this sort of thing easier.

After 1979 there was an \acro{ANSI} standard for terminal control codes, based on the
\acro{DEC VT-100} (being supported in the \acro{IBM PC}'s original screen driver gave it a
boost). By the early 1990s \acro{ANSI} conformance was close to universal in \acro{VDT}s,
which is why that's what your software terminal emulator does.

This whole technology category was rapidly wiped out in general-purpose
computing, like dinosaurs after the Alvarez strike, when bit-mapped color
displays on personal computers that could match the dot pitch of a monochrome
\acro{VDT} became relatively inexpensive, around 1992. The legacy \acro{VDT} hardware
lingered longest in dedicated point-of-sale systems, remaining not uncommon
until as late as 2010 or so.

It's not true, as is sometimes suggested, that heritage from the \acro{VDT} era
explains the \UNIX/ command line --- that actually predated \acro{VDT}s, going back to the
last generation of printing terminals in the late 1960s and early 1970s. Every
hacker once knew that this is why we often speak of ``printing'' output when we
mean sending it to standard output that is normally connected to a terminal
emulator.

What the \acro{VDT} era \textbf{does} explain is some of our heritage games (see next section)
and a few surviving utility programs like \prog{vi(1)}, \prog{top(1)} and \prog{mutt(1)}. These are
what advanced visual interfaces looked like in the \acro{VDT} era, before bitmapped
displays and \acro{GUI}s. This is why program interfaces that are two-dimensional but
use characters only are now called \acro{TUI} (``terminal user interface''), but the
term is an anachronism; it was coined after ``\acro{GUI}'' became common.

It also explains ``screensavers.'' Every hacker once knew that these are so
called because cathode-ray tubes were subject to a phenomenon called ``phosphor
burn-in'' that could permanently damage the screen's ability to display
information. Software to randomly vary the image(s) on your display was
originally written to prevent burn-in. Flatscreens don't have this problem; the
secondary purpose of doing something visually interesting with that blank space
took over.

\section{The Early, Awful Days of Bitmapped Displays}
Terminals --- which could usually only display a fixed alphabet of formed
characters --- were eventually replaced by the ``bitmapped'' style of display we're
used to today, with individual screen pixels manipulable. Every hacker once
knew that though there were earlier precedents done as research, the first
production system with a bitmapped display was the Alto, built at the Xerox
Palo Alto Research Center in 1973. (The laser printer and Ethernet were also
invented there.)

It was \strong{not} generally known that the Alto had a display only 608 pixels wide and
808 high --- folk memory confused it with later displays in the 1024$\times$1024 range.
Its 1981 successor the Dandelion achieved 1024$\times$809; an attempt to commercialize
it as the ``Xerox Star'' failed. But in 1982 the newly-born Sun Microsystems
shipped the Sun-1 with 1024$\times$800 pixels; descendants of this machine (and very
similar designs by other vendors) became enormously successful --- until they
were wiped out by descendants of the \acro{IBM PC} in the late 1990s. Before that,
most hackers dreamed of owning a Sun-class workstation, and the bitmapped
display was a significant part of that allure.

Alas, these machines were too expensive for individuals. But from 1975 onwards
primitive bitmapped-display began to appear on personal color computers such as
the Apple II, often through consumer-grade television sets. These had one
feature the workstation displays of the time lacked --- color --- but their pixel
resolutions were laughable by comparison. The Apple II in ``Hi-res'' mode could
only manage 280$\times$192, an area you could easily cover with the palm of your hand
on the displays of 2017.

People who remember these early consumer-grade color displays show a tendency
to both think they arrived sooner than they did and filter their memories
through a nostalgic haze. In reality they remained astoundingly bad compared to
even a Sun-1 for a long time after the Apple II; you couldn't get even near the
crispness of an 80$\times$24 display of text on a \acro{VDT} with them. Using a
consumer-grade TV also meant coping with eyestrain-inducing artifacts around
the limited amount of text it could display, due to \acro{NTSC} chroma-luminance
crossover.

Because hackers \strong{needed} good text display for programming --- and because many
knew what workstation graphics in the 1024$\times$1024 range looked like --- many of us
dismissed these displays (and the computers they were attached to) as
near-useless toys. We sought workstation graphics when we could get it and used
\acro{VDT}s when we couldn't.

In 1984, the original Macintosh was the first consumer-grade machine with a
dedicated bit-mapped display that come within even distant hail of
workstation-class graphics --- 512$\times$342 black and white. It was shortly followed
by the \acro{IBM EGA} adapter and its numerous cheap clones with 640$\times$350 at a whopping
16 colors, accompanied by flicker and other artifacts. In the following years,
256 colors gradually crept in, starting with a resolution of --- wait for it --- 
320$\times$200.

For another five years or so consumer hardware was still split between
low-resolution b\&w displays and abysmally low-resolution, hard-to-use color
displays. If we had to use a consumer-grade computer at all, most of us tended
to prefer the many b\&w displays that were relatively inexpensive, surprisingly
nice on the eyes and better for our life's work --- code.

There was, however, a cohort of hackers (especially among the younger and newer
ones of the mid-1980s) who worked hard to get as much performance as they could
from the pathetic color displays and \acro{TV}s, as well as from the earliest attempts
at computer sound hardware. The resulting chunky graphics with pixelated
``sprites'' moving across the screen, accompanied by a chirpy/buzzy style of
music and sound effects is still used in some 21st-century games for pure
nostalgic value. If you hear a reference to ``8-bit'' graphics or sound, that's
what it means in these latter days.

Monitors capable of 1024$\times$1024 color display did not reach even the high end of
the consumer market until about 1990 and did not become generally available
until about 1992. Not by coincidence, this was when \acro{PC} manufacturers began to
put serious pressure on workstation vendors. But for some years after that the
quality of these displays remained relatively poor, with coarse dot pitches and
fringing effects due to the expense and difficulty of manufacturing decent
color masks for cathode-ray tubes. These problems weren't fully solved until
\acro{CRT}s were on the verge of being obsolesced by flatscreens.

Once we were able to take color displays at 1024$\times$1024 and up for granted, a lot
of this history faded from memory. Even people who lived through it have a
tendency to forget what conditions were like before flatscreens, and are often
surprised to be reminded how low the pixel resolutions were on older hardware.

\section{Games Before GUIs}
Before bit-mapped color displays became common and made graphics-intensive
games the norm, there was a vigorous tradition of games that required only
textual interfaces or the character-cell graphics on a \acro{VDT}.

These \acro{VDT} games often found their way to early microcomputers as well. In part
this was because some of those early micros themselves had weak or nonexistent
graphical capabilities, and in part because textual games were relatively easy
to port and featured as type-in projects in magazines and books.

The oldest group of games that were once common knowledge are the Trek family,
a clade of games going back to 1971 in which the player flew the starship
Enterprise through the Federation fighting Klingons and Romulans and other
enemies. Every hacker over a certain age remembers spending hours playing
these.

The history of the Trek clade is too complex to summarize here. The thing to
notice about them is that the extremely crude interface (designed not even for
\acro{VDT}s but for teletypes!) hid what was actually a relatively sophisticated
wargame in which initiative, tactical surprise, and logistics all played
significant roles.

Every hacker once knew what the phrase ``You are in a maze of twisty little
passages, all alike'' meant, and often used variants about confusing situations
in real life (For example, ``You are in a maze of twisty little technical
standards, all different''). It was from the very first dungeon-crawling
adventure game, Colossal Cave Adventure (1976--77). People who knew this game
from its beginnings often thought of it as \acro{ADVENT}, after its 6-character
filename on the \acro{PDP-10} where it first ran.

When the original author of \acro{ADVENT} wasn't inventing an entire genre of computer
games, he was writing the low-level firmware for some of the earliest \acro{ARPANET}
routers. This was not generally known at the time, but illustrates how intimate
the connection between these early games and the cutting edge of serious
programming was. ``Game designer'' was not yet a separate thing, then.

You might occasionally encounter ``xyzzy'' as a nonce variable name. Every hacker
used to know that xyzzy was a magic word in \acro{ADVENT}.

\acro{ADVENT} had a direct successor that was even more popular --- Zork, first released
in 1979 by hackers at \acro{MIT} on a \acro{PDP-10} (initially under the name ``Dungeon'') and
later successfully commercialized. This game is why every hacker once knew that
a zorkmid was the currency of the Great Underground Empire, and that if you
wander around in dark places without your lantern lit you might be eaten by a
grue.

There was another family of games that took a different, more visual approach
to dungeon-crawling. They are generally called ``roguelikes'', after the earliest
widely-distributed games in this group, Rogue from 1980. They featured
top-down, maplike views of dungeon levels through which the player would wander
battling monsters and seeking treasure.

The most widely played games in this group were Hack (1982) and Nethack (1987).
Nethack is notable for having been one of the earliest programs in which the
development group was consciously organized as a distributed collaboration over
the Internet; at the time, this was a sufficiently novel idea to be advertised
in the project's name.

Rogue's descendants were the most popular and successful \acro{TUI} games ever. Though
they gradually passed out of universal common knowledge after the mid-1990s,
they retain devoted minority followings even today. Their fans accurately point
out that the primitive state of interface design encouraged concentration on
plot and story values, leading to a surprisingly rich imaginative experience.

\section{ASCII}
\ASCII/, the American Standard Code for Information Interchange, evolved in the
early 1960s out of a family of character codes used on teletypes.

\ASCII/, unlike a lot of other early character encodings, is likely to live
forever --- because by design the low 127 code points of Unicode \strong{are} \ASCII/. If
you know what \acro{UTF-8} is (and you should) every \ASCII/ file is correct \acro{UTF-8} as
well.

The following table describes \ASCII/-1967, the version in use today. This is the
16$\times$8 format given in most references.

\begin{center}
\leavevmode\vbox{\footnotesize\ttfamily\offinterlineskip
\halign{%
\hfil#&\hskip0.5em\hfil#&\hskip0.5em#\hfil&\hskip0.38em\vrule height1.5ex depth1ex \hskip0.38em%
\hfil#&\hskip0.5em\hfil#&\hskip0.5em#\hfil&\hskip0.38em\vrule\hskip0.38em%
\hfil#&\hskip0.5em\hfil#&\hskip0.5em#\hfil&\hskip0.38em\vrule\hskip0.38em%
\hfil#&\hskip0.5em\hfil#&\hskip0.5em#\hfil&\hskip0.38em\vrule\hskip0.38em%
\hfil#&\hskip0.5em\hfil#&\hskip0.5em#\hfil&\hskip0.38em\vrule\hskip0.38em%
\hfil#&\hskip0.5em\hfil#&\hskip0.5em#\hfil&\hskip0.38em\vrule\hskip0.38em%
\hfil#&\hskip0.5em\hfil#&\hskip0.5em#\hfil&\hskip0.38em\vrule\hskip0.38em%
\hfil#&\hskip0.5em\hfil#&\hskip0.5em#\hfil\cr
\omit\hfil Dec&\omit\ Hex&&%
\omit\hskip 0.38em\vrule height1.5ex depth1ex\hskip 0.38em\hfil Dec&\omit\ Hex&&%
\omit\hskip 0.38em\vrule\hskip 0.38em\hfil Dec&\omit\ Hex&&%
\omit\hskip 0.38em\vrule\hskip 0.38em\hfil Dec&\omit\ Hex&&%
\omit\hskip 0.38em\vrule\hskip 0.38em\hfil Dec&\omit\ Hex&&%
\omit\hskip 0.38em\vrule\hskip 0.38em\hfil Dec&\omit\ Hex&&%
\omit\hskip 0.38em\vrule\hskip 0.38em\hfil Dec&\omit\ Hex&&%
\omit\hskip 0.38em\vrule\hskip 0.38em\hfil Dec&\omit\ Hex&\cr
0&00&NUL&16&10&DLE&32&20& &48&30&0&64&40&@&80&50&P&96&60&`&112&70&p\cr
1&01&SOH&17&11&DC1&33&21&!&49&31&1&65&41&A&81&51&Q&97&61&a&113&71&q\cr
2&02&STX&18&12&DC2&34&22&"&50&32&2&66&42&B&82&52&R&98&62&b&114&72&r\cr
3&03&ETX&19&13&DC3&35&23&\#&51&33&3&67&43&C&83&53&S&99&63&c&115&73&s\cr
4&04&EOT&20&14&DC4&36&24&\$&52&34&4&68&44&D&84&54&T&100&64&d&116&74&t\cr
5&05&ENQ&21&15&NAK&37&25&\%&53&35&5&69&45&E&85&55&U&101&65&e&117&75&u\cr
6&06&ACK&22&16&SYN&38&26&\&&54&36&6&70&46&F&86&56&V&102&66&f&118&76&v\cr
7&07&BEL&23&17&ETB&39&27&'&55&37&7&71&47&G&87&57&W&103&67&g&119&77&w\cr
8&08&BS&24&18&CAN&40&28&(&56&38&8&72&48&H&88&58&X&104&68&h&120&78&x\cr
9&09&HT&25&19&EM&41&29&)&57&39&9&73&49&I&89&59&Y&105&69&i&121&79&y\cr
10&0A&LF&26&1A&SUB&42&2A&*&58&3A&:&74&4A&J&90&5A&Z&106&6A&j&122&7A&z\cr
11&0B&VT&27&1B&ESC&43&2B&+&59&3B&;&75&4B&K&91&5B&[&107&6B&k&123&7B&\char`\{\cr
12&0C&FF&28&1C&FS&44&2C&,&60&3C&<&76&4C&L&92&5C&\char`\\&108&6C&l&124&7C&|\cr
13&0D&CR&29&1D&GS&45&2D&-&61&3D&=&77&4D&M&93&5D&]&109&6D&m&125&7D&\char`\}\cr
14&0E&SO&30&1E&RS&46&2E&.&62&3E&>&78&4E&N&94&5E&\char`^&110&6E&n&126&7E&\char`\~\cr
15&0F&SI&31&1F&US&47&2F&/&63&3F&?&79&4F&O&95&5F&\char`_&111&6F&o&127&7F&DEL\cr
}}
\end{center}

\noindent However, this format --- less used because the shape is inconvenient --- probably
does more to explain the encoding:

\begin{center}
\leavevmode\vbox{\footnotesize\ttfamily\offinterlineskip
\halign{%
#\hfil&\quad#\hfil&\hskip1.5em\vrule height 1.5ex depth1ex\hskip1.5em#\hfil&\quad#\hfil%
&\hskip1.5em\vrule\hskip1.5em#\hfil&\quad#\hfil%
&\hskip1.5em\vrule\hskip1.5em#\hfil&\quad#\hfil\cr
0000000&NUL&0100000&&1000000&@&1100000&`\cr
0000001&SOH&0100001&!&1000001&A&1100001&a\cr
0000010&STX&0100010&"&1000010&B&1100010&b\cr
0000011&ETX&0100011&\#&1000011&C&1100011&c\cr
0000100&EOT&0100100&\$&1000100&D&1100100&d\cr
0000101&ENQ&0100101&\%&1000101&E&1100101&e\cr
0000110&ACK&0100110&\&&1000110&F&1100110&f\cr
0000111&BEL&0100111&'&1000111&G&1100111&g\cr
0001000&BS&0101000&(&1001000&H&1101000&h\cr
0001001&HT&0101001&)&1001001&I&1101001&i\cr
0001010&LF&0101010&*&1001010&J&1101010&j\cr
0001011&VT&0101011&+&1001011&K&1101011&k\cr
0001100&FF&0101100&,&1001100&L&1101100&l\cr
0001101&CR&0101101&-&1001101&M&1101101&m\cr
0001110&SO&0101110&.&1001110&N&1101110&n\cr
0001111&SI&0101111&/&1001111&O&1101111&o\cr
0010000&DLE&0110000&0&1010000&P&1110000&p\cr
0010001&DC1&0110001&1&1010001&Q&1110001&q\cr
0010010&DC2&0110010&2&1010010&R&1110010&r\cr
0010011&DC3&0110011&3&1010011&S&1110011&s\cr
0010100&DC4&0110100&4&1010100&T&1110100&t\cr
0010101&NAK&0110101&5&1010101&U&1110101&u\cr
0010110&SYN&0110110&6&1010110&V&1110110&v\cr
0010111&ETB&0110111&7&1010111&W&1110111&w\cr
0011000&CAN&0111000&8&1011000&X&1111000&x\cr
0011001&EM&0111001&9&1011001&Y&1111001&y\cr
0011010&SUB&0111010&:&1011010&Z&1111010&z\cr
0011011&ESC&0111011&;&1011011&[&1111011&\char`\{\cr
0011100&FS&0111100&<&1011100&\char`\\&1111100&|\cr
0011101&GS&0111101&=&1011101&]&1111101&\char`\}\cr
0011110&RS&0111110&>&1011110&\char`^&1111110&\char`\~\cr
0011111&US&0111111&?&1011111&\char`_&1111111&DEL\cr
}}
\end{center}

\noindent Using the second table, it's easier to understand a couple of things:
\begin{itemize}
\item The Control modifier on your keyboard basically clears the top
three bits of whatever character you type, leaving the bottom five and mapping
it to the 0..31 range. So, for example, Ctrl-\acro{SPACE}, Ctrl-@, and Ctrl-` all mean
the same thing: \acro{NUL}.

\item Very old keyboards used to do Shift just by toggling the 32 or
16 bit, depending on the key; this is why the relationship between small and
capital letters in \ASCII/ is so regular, and the relationship between numbers
and symbols, and some pairs of symbols, is sort of regular if you squint at it.
The \acro{ASR-33}, which was an all-uppercase terminal, even let you generate some
punctuation characters it didn't have keys for by shifting the 16 bit; thus,
for example, Shift-K (\hexnum{0x4B}) became a \texttt{\char`\[} (\hexnum{0x5B}).
\end{itemize}
It used to be common knowledge that the original 1963 \ASCII/ had been
slightly different. It lacked tilde and vertical bar; \hexnum{5E} was an up-arrow rather
than a caret, and \hexnum{5F} was a left arrow rather than underscore. Some early
adopters (notably \acro{DEC}) held to the 1963 version.

If you learned your chops after 1990 or so, the mysterious part of this is
likely the control characters, code points 0--31. You probably know that \CEE/ uses
\acro{NUL} as a string terminator. Others, notably \acro{LF} = Line Feed and \acro{HT} = Horizontal
Tab, show up in plain text. But what about the rest?

Many of these are remnants from teletype protocols that have either been dead
for a very long time or, if still live, are completely unknown in computing
circles. A few had conventional meanings that were half-forgotten even before
Internet times. A \strong{very} few are still used in binary data protocols today.

Here's a tour of the meanings these had in older computing, or retain today. If
you feel an urge to send me more, remember that the emphasis here is on what
was common knowledge back in the day. If I don't know it now, we probably
didn't generally know it then.
\begin{description}
\item[\acro{NUL} (Null) = Ctrl-@] Survives as the string terminator in \CEE/.
\item[\acro{SOH} (Start of Heading) = Ctrl-A] Rarely used (as Ctrl-A) as a section
divider in otherwise textual formats. Some versions of \UNIX/ mailbox format used
it as a message divider. One very old version-control system (\acro{SCCS}) did
something similar.
\item[\acro{STX} (Start of Text), \acro{ETX} (End of Text) = Ctrl-B, Ctrl-C] Very rarely
used as packet or control-sequence delimiters. You will probably never see
this, and the only place I've ever seen it was on a non-\UNIX/ OS in the early
1980s. \acro{ETX} is Ctrl-C, which is a \acro{SIGINT} interrupt character on \UNIX/ systems,
but that has nothing to do with its \ASCII/ meaning per se and probably derives
from abbreviating the word ``Cancel.''
\item[\acro{EOT} (End of Transmission) = Ctrl-D] As Ctrl-D, the way you type ``End of
file'' to a \UNIX/ terminal.
\item[\acro{ENQ} (Enquiry) = Ctrl-E] In the days of hardware serial terminals, there
was a convention that if a computer sent \acro{ENQ} to a terminal, it should answer
back with terminal type identification. While this was not universal, it at
least gave computers a fighting chance of autoconfiguring what capabilities it
could assume the terminal to have. Further back, on teletypes, the answerback
had been a station ID rather than a device type; as late as the 1970s it was
still generally remembered that \acro{ENQ}'s earliest name in \ASCII/ had been \acro{WRU} (``Who
are you?'').
\item[\acro{ACK} (Acknowledge) = Ctrl-F]
It used to be common for wire protocols written in \ASCII/ to use \acro{ENQ/ACK} as a
handshake, sometimes with \acro{NAK} as a failure indication (the \acro{XMODEM}/\acro{YMODEM}/\acro{ZMODEM}
protocol did this). Hackers used to use \acro{ACK} in speech as ``I hear you'' and
were a bit put out when this convention was disrupted in the 1980s by Bill The
Cat's ``Ack! Thppt!''
\item[\acro{BEL} (Bell) = Ctrl-G]
Make the bell ring on the teletype --- an attention signal. This often worked on
\acro{VDT}s as well, but is no longer reliably the default on software terminal
emulators. Some map it to a visual indication like flashing the title bar.
\item[\acro{BS} (Backspace) = Ctrl-H]
Still does what it says on the tin, though there has been some historical
confusion over whether the backspace key on a keyboard should behave like \acro{BS}
(nondestructive cursor move) or \acro{DEL} (backspace and delete). Never used in
textual data protocols.
\item[\acro{HT} (Horizontal tab) = Ctrl-I]
Still does what it says on the tin. Sometimes used as a field separator in \UNIX/
textual file formats, but this is now old-fashioned and declining in usage.
\item[\acro{LF} (Line Feed) = Ctrl-J]
The \UNIX/ textual end-of-line. Printing terminals interpreted it as ``scroll down
one line''; the \UNIX/ tty driver would normally wedge in a \acro{CR} right before it on
output (or in early versions, right after).
\item[\acro{VT} (Vertical Tab) = Ctrl-K]
In the days of printing terminals this often caused them to scroll down a
configurable number of lines. \acro{VDT}s had any number of possible behaviors; at
least some pre-\acro{ANSI} ones interpreted \acro{VT} as ``scroll \strong{up} one line.'' The only
reason anybody remembers this one at all is that it persisted in \UNIX/
definitions of what a whitespace character is, even though it's now extinct in
the wild.
\item[\acro{FF} (Form Feed) = Ctrl-L] Eject the current page from your printing
terminal. Many \acro{VDT}s interpreted this as a ``clear screen'' instruction. Software
terminal emulators sometimes still do. Often interpreted as a ``screen refresh''
request in textual-input \UNIX/ programs that bind other control characters
(shells, editors, more/less, etc)
\item[\acro{CR} (Carriage Return) = Ctrl-M]
It is now possible that the reader has never seen a typewriter, so this needs
explanation: ``carriage return'' is the operation of moving your print head or
cursor to the left margin. Windows, other non-\UNIX/ operating systems, and some
Internet protocols (such as \acro{SMTP}) tend to use \acro{CR-LF} as a line terminator,
rather than bare \acro{LF}. The reason it was \acro{CR-LF} rather than \acro{LF-CR} goes back to
Teletypes: a Teletype printed ten characters per second, but the print-head
carriage took longer than a tenth of a second to return to the left side of the
paper. So if you ended a line with line-feed, then carriage-return, you would
usually see the first character of the next line smeared across the middle of
the paper, having been struck while the carriage was still zipping to the left.
Pre-\UNIX/ MacOS used a bare \acro{CR}.
\item[\acro{SO} (Shift Out), \acro{SI} (Shift In) = Ctrl-N, Ctrl-O]
Escapes to and from an alternate character set. \UNIX/ software used to emit them
to drive pre-\acro{ANSI} \acro{VDT}s that interpreted them that way, but native \UNIX/ usage is
rare to nonexistent. On teletypes with a two-color ink ribbon (the second color
usually being red) \acro{SO} was a command to shift to the alternate color, \acro{SI} to
shift back.
\item[\acro{DLE} (Data Link Escape) = Ctrl-P]
Sometimes used as a packet-framing character in binary protocols. That is, a
packet starts with a \acro{DLE}, ends with a \acro{DLE}, and if one of the interior data
bytes matches \acro{DLE} it is doubled.
\item[\acro{DC[1234]} (Device Control [1234]) = Ctrl-\acro{[QRST]}]
Never to my knowledge used specially after teletypes. However: there was a
common software flow-control protocol, used over \ASCII/ but separate from it, in
which \acro{XOFF} (\acro{DC3}) was used as a request to pause transmission and \acro{XON} (\acro{DC1}) was
used as a request to resume transmission. As Ctrl-S and Ctrl-Q these were
implemented in the \UNIX/ terminal driver and long outlived their origin in the
Model 33 Teletype. And not just \UNIX/; this was implemented in \acro{CP/M} and \acro{DOS},
too.
\item[\acro{NAK} (Negative Acknowledge) = Ctrl-U] See the discussion of \acro{ACK} above.
\item[\acro{SYN} (Synchronous Idle) = Ctrl-V]
Never to my knowledge used specially after teletypes, except in synchronous
serial protocols never used on micros or minis. Be careful not to confuse this
with the \acro{SYN} (synchronization) packet used in \acro{TCP/IP}'s \acro{SYN} \acro{SYN-ACK}
initialization sequence. In an unrelated usage, many \UNIX/ tty drivers use this
(as Ctrl-V) for the literal-next character that lets you quote following
control characters such as Ctrl-C.
\item[\acro{ETB} (End of Transmission Block) = Ctrl-W]
Nowadays this is usually ``kill window'' on a web browser, but it used to mean
``delete previous word'' in some contexts and sometimes still does.
\item[\acro{CAN} (Cancel), \acro{EM} (End of Medium) = Ctrl-X, Ctrl-Y]
Never to my knowledge used specially after teletypes.
\item[\acro{SUB} (Substitute) = Ctrl-Z]
\acro{DOS} and Windows use Ctrl-Z (\acro{SUB)} as an end-of-file character; this is unrelated
to its \ASCII/ meaning. It was common knowledge then that this use of $^\wedge$Z
had been inherited from a now largely forgotten earlier OS called \acro{CP/M} (1974),
and into \acro{CP/M} from earlier \acro{DEC} minicomputer OSes such as \acro{RSX-11} (1972). \UNIX/
uses Ctrl-Z as the ``suspend process'' command keystroke.
\item[\acro{ESC} (Escape)]
Still commonly used as a control-sequence introducer. This usage is especially
associated with the control sequences recognized by \acro{VT100} and \acro{ANSI}-standard
\acro{VDT}s, and today by essentially all software terminal emulators
\item[\acro{[FGRU]S} ($\lbrace$Field$\vert$Group$\vert$Record$\vert$Unit$\rbrace$ Separator)]
There are some uses of these in \acro{ATM} and bank protocols (these have never been
common knowledge, but I'm adding this note to forestall yet more repetitions
from area specialists who will apparently otherwise keep telling me about it
until the end of time). \acro{FS}, as Ctrl-$\backslash$, sends \acro{SIGQUIT} under some
\UNIX/es, but this has nothing to do with \ASCII/. Ctrl-] (\acro{GS}) is the exit
character from telnet, but this also has nothing to do with its \ASCII/ meaning.
\item[\acro{DEL} (Delete)]
Usually an input character meaning ``backspace and delete.'' Under older \UNIX/
variants, sometimes a \acro{SIGINT} interrupt character.
\end{description}
\noindent Not all of these were so well known that any hacker could instantly
map from mnemonic to binary, or vice-versa. The well-known set was roughly \acro{NUL},
\acro{BEL}, \acro{BS}, \acro{HT}, \acro{LF}, \acro{FF}, \acro{CR}, \acro{ESC}, and \acro{DEL}.

There are a few other bits of \ASCII/ lore worth keeping in mind $\ldots$
\begin{itemize}
\item A Meta or Alt key on a \acro{VDT} added 128 to the \ASCII/ keycode for
whatever it's modifying (probably --- on a few machines with peculiar word
lengths they did different things). Software terminal emulators have more
variable behavior; many of them now simply insert an \acro{ESC} before the modified
key, which Emacs treats as equivalent to adding 128.

\item An item of once-common knowledge that was half-forgotten fairly early (like,
soon after \acro{VDT}s replaced teletypes) is that the binary value of \acro{DEL} (\hexnum{0x7F},
\hexnum{0b01111111}) descends from its use on paper tape. Seven punches could overwrite
any character in \ASCII/, and tape readers skipped \acro{DEL} (and \acro{NUL} --- no punches).
This is why \acro{DEL} was anciently called the ``Rubout'' character and is an island at
the other end of the \ASCII/ table from the other control characters.

\item \acro{VDT} keyboards often had a ``Break'' key inherited from the \acro{ASR-33} (there's a
vestigial remnant of this even on the \acro{IBM PC} keyboard). This didn't send a
well-formed \ASCII/ character; rather, it caused an out-of-band condition that
would be seen as a \acro{NUL} with a framing error at the other end. This was used as
an attention or interrupt signal.
\end{itemize}
\noindent You can study the bit structure of \ASCII/ using \prog{ascii(1)}. Both of the tables
above were generated using it.

\section{The Slow Birth of Distributed Collaboration}
Nowadays we take for granted a public infrastructure of distributed version
control and a lot of practices for distributed teamwork that go with it -
including development teams that never physically have to meet. But these
tools, and awareness of how to use them, were a long time developing. They
replace whole layers of earlier practices that were once general but are now
half- or entirely forgotten.

The earliest practice I can identify that was directly ancestral was the \acro{DECUS}
tapes. \acro{DECUS} was the Digital Equipment Corporation Users' Group, chartered in
1961. One of its principal activities was circulating magnetic tapes of
public-domain software shared by \acro{DEC} users. The early history of these tapes is
not well-documented, but the habit was well in place by 1976.

One trace of the \acro{DECUS} tapes seems to be the \acro{README} convention. While it
entered the \UNIX/ world through \acro{USENET} in the early 1980s, it seems to have
spread there from \acro{DECUS} tapes. The \acro{DECUS} tapes begat the \acro{USENET} source-code
groups, which were the incubator of the practices that later became ``open
source.'' \UNIX/ hackers used to watch for interesting new stuff on
\texttt{comp.sources.unix} as automatically as they drank their morning coffee.

The \acro{DECUS} tapes and the \acro{USENET} sources groups were more of a publishing channel
than a collaboration medium, though. Three pieces were missing to fully support
that: version control, patching, and forges.

Version control was born in 1972, though \acro{SCCS} (Source Code Control System)
didn't escape Bell Labs until 1977. The proprietary licensing of \acro{SCCS} slowed
its uptake; one response was the freely reusable \acro{RCS} (Revision Control System)
in 1982.

The first real step towards across-network collaboration was the \prog{patch(1)}
utility in 1984. The concept seems so obvious now that even hackers who predate
\prog{patch(1)} have trouble remembering what it was like when we only knew how to
pass around source-code changes as entire altered files. But that's how it was.

Even with \acro{SCCS}/\acro{RCS}/patch the friction costs of distributed development over the
Internet were still so high that some years passed before anyone thought to try
it seriously. I have looked for, but not found, definite examples earlier than
nethack. This was a roguelike game launched in 1987. Nethack developers passed
around whole files --- and later patches --- by email, sometimes using \acro{SCCS} or \acro{RCS}
to manage local copies.

Distributed development could not really get going until the third major step
in version control. That was \acro{CVS} (Concurrent Version System) in 1990, the
oldest \acro{VCS} still in wide use at time of writing in 2017. Though obsolete and
now half-forgotten, \acro{CVS} was the first version-control system to become so
ubiquitous that every hacker once knew it. \acro{CVS}, however, had significant design
flaws; it fell out of use rapidly when better alternatives became
available.

Between around 1989 and the breakout of mass-market Internet in 1993--1994, fast
Internet became available enough to hackers that distributed development in the
modern style began to become thinkable. The next major steps were not technical
changes but cultural ones.

In 1991 Linus Torvalds announced Linux as a distributed collaborative effort.
It is now easy to forget that early Linux development used the same
patch-by-email method as nethack --- there were no public Linux repositories yet.
The idea that there \strong{ought} to be public repositories as a normal practice for
major projects (in addition to shipping source tarballs) wouldn't really take
hold until after I published ``The Cathedral and the Bazaar'' in 1997. While CatB
was influential in promoting distributed development via shared public
repositories, the technical weaknesses of \acro{CVS} were in hindsight probably an
equally important reason this practice did not become established sooner and
faster.

The first dedicated software forge was not spun up until 1999. That was
SourceForge, still extant today (2018). At first it supported only \acro{CVS}, but it
sped up the adoption of the (greatly superior) Subversion, launched in 2000 by
a group of former \acro{CVS} developers.

Between 2000 and 2005 Subversion became ubiquitous common knowledge. But in
2005 Linus Torvalds invented git, which would fairly rapidly obsolesce all
previous version-control systems and is a thing every hacker \strong{now} knows.

\section{Key Dates}
These are dates that every hacker knew were important at the time, or shortly
afterwards. I've tried to concentrate on milestones for which the date --- or the
milestone itself --- seems to have later passed out of folk memory.
\begin{description}
\item[1961] \acro{MIT} takes delivery of a \acro{PDP-1}. The first recognizable ancestor of the hacker
culture of today rapidly coalesces around it.

\item[1969] Ken Thompson begins work on what will become \UNIX/. First commercial \acro{VDT} ships;
it's a glass \acro{TTY}. First packets exchanged on the \acro{ARPANET}, the direct ancestor
of today's Internet.

\item[1970] \acro{DEC PDP-11} first ships; architectural descendants of this machine, including
later Intel microprocessors, will come to dominate computing.

\item[1973] Interdata 32 ships; the long 32-bit era begins. \UNIX/ Edition 5 (not yet on
the Interdata) escapes Bell Labs to take root at a number of educational
institutions. The \acro{XEROX} Alto pioneers the ``workstation'' --- a networked personal
computer with a high-resolution display and a mouse.

\item[1974] \acro{CP/M} first ships; this will be the OS for a large range of microcomputers until
effectively wiped out by \acro{MS-DOS} after 1981. \acro{MS-DOS} will, however, have been
largely cloned from \acro{CP/M}; this theft leaves rather unmistakable traces in the
\acro{BIOS}. It's also why \acro{MS-DOS} has filenames with at most 8 characters of name and
3 of extension.

\item[1975] First Altair 8800 ships; beginning of heroic age of microcomputers. First 24$\times$80
and 25$\times$80 ``smart'' (addressable-cursor) \acro{VDTs}. \acro{ARPANET} declared ``operational'',
begins to spread to major universities.

\item[1976] ``Lions' Commentary on \UNIX/ 6th Edition, with Source Code'' released. First look
into the \UNIX/ kernel source for most hackers, and was a huge deal in those
pre-open-source days. First version of \acro{ADVENT} is written. First version of the
Emacs text editor.

\item[1977] \UNIX/ ported to the Interdata; first version with a kernel written largely in C
rather than machine-dependent assembler. Second generation of home computers
(Apple II and \acro{TRS-80} Model 1) ship. \acro{SCCS}, the first version-control system, is
publicly released.

\item[1978] First \acro{BBS} launched --- \acro{CBBS}, in Chicago.

\item[1979] \acro{MIT} Dungeon, later known as Zork, is written; the first grues lurk in dark
places.

\item[1980] Rogue, ancestral to all later top-view dungeon-crawling games, is invented.
\acro{USENET} begins.

\item[1981] First \acro{IBM PC} ships; end of the heroic age of micros. \acro{TCP/IP} is implemented on a
\acro{VAX-11/780} under 4.1\acro{BSD} \UNIX/; \acro{ARPANET} and \UNIX/ cultures begin to merge.

\item[1982] After some false starts from 1980--1981 with earlier 68000-based micros of
similar design, the era of commercial \UNIX/ workstations truly begins with the
founding and early success of Sun Microsystems. \acro{RCS}, the second version-control
system, ships.

\item[1983] \acro{PDP-10} canceled; this is effectively the end of 36-bit architectures anywhere
outside of deep mainframe country, though Symbolics Lisp machines hold out a
while longer. \acro{ARPANET}, undergoing some significant technical changes, becomes
Internet.

\item[1984] \acro{AT\&T} begins a largely botched attempt to commercialize \UNIX/, clamping down on
access to source code. In the \acro{BBS} world, FidoNet is invented. The \prog{patch(1)}
utility is invented.

\item[1985] RMS published \acro{GNU} Manifesto. This is also roughly the year the \CEE/ language
became the dominant lingua franca of both systems and applications programming,
eventually displacing earlier compiled language so completely that they are
almost forgotten. First Model M keyboard ships.

\item[1986] Intel 386 ships; end of the line for 8- and 16-bit \acro{PC}s. Consumer-grade hardware
in this class wouldn't be generally available until around 1989, but after that
would rapidly surpass earlier 32-bit minicomputers and workstations in
capability.

\item[1987] \acro{USENET} undergoes the Great Renaming. Perl, first of the modern scripting
languages, is invented.

\item[1991] Linux and the World Wide Web are (separately) launched. Python scripting
language invented.

\item[1992] Bit-mapped color displays with a dot pitch matching that of a monochrome \acro{VDT}
(and a matching ability to display crisp text at 80$\times$25) ship on consumer-grade
\acro{PC}s. Bottom falls out of the \acro{VDT} market.

\item[1993] Linux gets \acro{TCP/IP} capability, moves from hobbyist's toy to serious OS. America
OnLine offers \acro{USENET} access to its users; ``September That Never Ended'' begins.
Mosaic adds graphics and image capability to the World Wide Web.

\item[1994] Mass-market Internet takes off in the U.S\hbox{}. \acro{USB} promulgated.

\item[1995--1996] Peak years of \acro{UUCP/USENET} and the \acro{BBS} culture, then collapse under pressure
from mass-market Internet.

\item[1997] I first give the ``Cathedral and Bazaar'' talk.

\item[1999] Banner year of the dot-com bubble. End of workstation era: Market for Suns and
other proprietary \UNIX/ workstations collapses under pressure from Linux running
on \acro{PCs}. Launch of SourceForge, the first public shared-repository site.

\item[2000] Subversion first ships.

\item[2001] Dot-com bubble pops. \acro{PC} hardware with workstation-class capabilities becomes
fully commoditized; pace of visible innovation in mass-market computers slows
noticeably.

\item[2005] Major manufacturers cease production of cathode-ray tubes in favor of
flat-panel displays. Flat-panels have been ubiquitous on new hardware since
about 2003. There is a brief window until about 2007 during which high-end \acro{CRT}s
no longer in production still exceed the resolution of flat-panel displays and
are still sought after. Also in 2005, \acro{AOL} drops \acro{USENET} support and Endless
September ends. Git first ships.

\item[2007--2008] 64-bit transition in mass market \acro{PC}s; the 32-bit era ends. Single-processor
speeds plateau at $4\pm0.25$GHz. iPhone and Android (both with \UNIX/-based OSes)
first ship.
\end{description}
\end{document}
