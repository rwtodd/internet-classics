% An early Java(TM) whitepaper
\newcount\chapcnt \chapcnt=0\newcount\seccnt \seccnt=0%
\newcount\subseccnt \subseccnt=0%
\def\tmk{$^{\rm TM}$}%
\def\Java{Java\tmk}\def\HotJava{HotJava\tmk}%
\def\chapter#1.{\seccnt=0\subseccnt=0\advance\chapcnt by1%
\bigbreak\centerline{\bf Chapter \the\chapcnt: #1}}%
\def\section#1.{\subseccnt=0\advance\seccnt by1%
\medbreak\noindent{\bf \the\chapcnt.\the\seccnt: #1}\par\noindent\ignorespaces}%
\def\subsection#1.{\advance\subseccnt by1%
\medbreak\noindent{\bf \the\chapcnt.\the\seccnt.\the\subseccnt: #1}\par\noindent\ignorespaces}%
\def\heading#1.{\medbreak\noindent{\bf #1}\par\noindent\ignorespaces}%
\def\bitem{\item{$\bullet$}}%
%
\centerline{\bf The Java Language Environment White Paper}
\centerline{May 1996}
\centerline{James Gosling}
\centerline{Henry McGilton}
\chapter Introduction to \Java~Technology.
{\medbreak\narrower\obeylines
The Next Stage of the Known,
Or a Completely New Paradigm?
\hskip 1in Taiichi Sakaiya --- {\sl The Knowledge-Value Revolution}\medbreak}

\heading The Software Developer's Burden. Imagine you're a software application
developer. Your programming language of choice (or the language that's been
foisted on you) is C or C++. You've been at this for quite a while and your job
doesn't seem to be getting any easier. These past few years you've seen the
growth of multiple incompatible hardware architectures, each supporting
multiple incompatible operating systems, with each platform operating with one
or more incompatible graphical user interfaces. Now you're supposed to cope
with all this and make your applications work in a distributed client-server
environment. The growth of the Internet, the World-Wide Web, and ``electronic
commerce'' have introduced new dimensions of complexity into the development
process.

The tools you use to develop applications don't seem to help you much. You're
still coping with the same old problems; the fashionable new object-oriented
techniques seem to have added new problems without solving the old ones. You
say to yourself and your friends, ``There has to be a better way!''

\heading The Better Way is Here Now. Now there is a better way --- the \Java{}
programming language platform from Sun Microsystems. Imagine, if you will, this
development world $\ldots$
{\medbreak\narrower
\bitem Your programming language is object oriented, yet it's still dead simple.
\bitem Your development cycle is much faster because Java technology is
interpreted. The compile-link-load-test-crash-debug cycle is obsolete---now you
just compile and run.
\bitem Your applications are portable across multiple platforms. Write your
applications once, and you never need to port them---they
will run without modification on multiple operating systems and hardware
architectures.
\bitem Your applications are robust because the Java runtime environment
manages memory for you.
\bitem Your interactive graphical applications have high performance because
multiple concurrent threads of activity in your application are supported by
the multithreading built into the Java programming language and runtime
platform.
\bitem Your applications are adaptable to changing environments because you can
dynamically download code modules from anywhere on the network.
\bitem Your end users can trust that your applications are secure, even though
they're downloading code from all over the Internet; the Java runtime
environment has built-in protection against viruses and tampering.
\medbreak}
\noindent You don't need to dream about these features. They're here now. The
Java programming language platform provides a portable, interpreted,
high-performance, simple, object-oriented programming language and supporting
run-time environment. This introductory chapter provides you with a brief look
at the main design goals of the Java system; the remainder of this paper
examines the features of Java in more detail.

The last chapter of this paper describes the \HotJava~Browser. The HotJava
Browser is an innovative World-Wide Web browser, and the first major
applications written using the Java platform. It is the first browser to
dynamically download and execute Java code fragments from anywhere on the
Internet, and can do so in a secure manner.

\section Beginnings of the \Java~Programming Language Project.
The \Java~programming language is designed to meet the challenges of
application development in the context of heterogeneous, network-wide
distributed environments. Paramount among these challenges is secure delivery
of applications that consume the minimum of system resources, can run on any
hardware and software platform, and can be extended dynamically.

The Java programming language originated as part of a research project to
develop advanced software for a wide variety of network devices and embedded
systems. The goal was to develop a small, reliable, portable, distributed,
real-time operating platform. When the project started, C++ was the language of
choice. But over time the difficulties encountered with C++ grew to the point
where the problems could best be addressed by creating an entirely new language
platform. Design and architecture decisions drew from a variety of languages
such as Eiffel, SmallTalk, Objective C, and Cedar/Mesa. The result is a
language platform that has proven ideal for developing secure, distributed,
network-based end-user applications in environments ranging from
network-embedded devices to the World-Wide Web and the desktop.

\section Design Goals of the \Java~Programming Language.
The design requirements of the JavaTM programming language are driven by the
nature of the computing environments in which software must be deployed. The
massive growth of the Internet and the World-Wide Web leads us to a completely
new way of looking at development and distribution of software. To live in the
world of electronic commerce and distribution, Java technology must enable the
development of secure, high performance, and highly robust applications on
multiple platforms in heterogeneous, distributed networks.

Operating on multiple platforms in heterogeneous networks invalidates the
traditional schemes of binary distribution, release, upgrade, patch, and so on.
To survive in this jungle, the Java programming language must be architecture
neutral, portable, and dynamically adaptable.

The system that emerged to meet these needs is simple, so it can be easily
programmed by most developers; familiar, so that current developers can easily
learn the Java programming language; object oriented, to take advantage of
modern software development methodologies and to fit into distributed
client-server applications; multithreaded, for high performance in applications
that need to perform multiple concurrent activities, such as multimedia; and
interpreted, for maximum portability and dynamic capabilities.

Together, the above requirements comprise quite a collection of buzzwords, so
let's examine some of them and their respective benefits before going on.

\subsection Simple, Object Oriented, and Familiar. Primary characteristics of
the Java programming language include a simple language that can be programmed
without extensive programmer training while being attuned to current software
practices. The fundamental concepts of Java technology are grasped quickly;
programmers can be productive from the very beginning.

The Java programming language is designed to be object oriented from the ground
up. Object technology has finally found its way into the programming mainstream
after a gestation period of thirty years. The needs of distributed,
client-server based systems coincide with the encapsulated, message-passing
paradigms of object-based software. To function within increasingly complex,
network-based environments, programming systems must adopt object-oriented
concepts. Java technology provides a clean and efficient object-based
development platform.

Programmers using the Java programming language can access existing libraries
of tested objects that provide functionality ranging from basic data types
through I/O and network interfaces to graphical user interface toolkits. These
libraries can be extended to provide new behavior.

Even though C++ was rejected as an implementation language, keeping the Java
programming language looking like C++ as far as possible results in it being a
familiar language, while removing the unnecessary complexities of C++. Having
the Java programming language retain many of the object-oriented features and
the "look and feel" of C++ means that programmers can migrate easily to the
Java platform and be productive quickly.


\subsection Robust and Secure. The Java programming language is designed for
creating highly reliable software. It provides extensive compile-time checking,
followed by a second level of run-time checking. Language features guide
programmers towards reliable programming habits.

The memory management model is extremely simple: objects are created with a {\tt new}
operator. There are no explicit programmer-defined pointer data types, no
pointer arithmetic, and automatic garbage collection. This simple memory
management model eliminates entire classes of programming errors that bedevil C
and C++ programmers. You can develop Java code with confidence that the system
will find many errors quickly and that major problems won't lay dormant until
after your production code has shipped.

Java technology is designed to operate in distributed environments, which means
that security is of paramount importance. With security features designed into
the language and run-time system, Java technology lets you construct
applications that can't be invaded from outside. In the network environment,
applications written in the Java programming language are secure from intrusion
by unauthorized code attempting to get behind the scenes and create viruses or
invade file systems.

\subsection Architecture Neutral and Portable.
Java technology is designed to support applications that will be deployed into
heterogeneous network environments. In such environments, applications must be
capable of executing on a variety of hardware architectures. Within this
variety of hardware platforms, applications must execute atop a variety of
operating systems and interoperate with multiple programming language
interfaces. To accommodate the diversity of operating environments, the Java
Compiler\tmk~product generates bytecodes---an architecture neutral intermediate
format designed to transport code efficiently to multiple hardware and software
platforms. The interpreted nature of Java technology solves both the binary
distribution problem and the version problem; the same Java programming
language byte codes will run on any platform.

Architecture neutrality is just one part of a truly portable system. Java
technology takes portability a stage further by being strict in its definition
of the basic language. Java technology puts a stake in the ground and specifies
the sizes of its basic data types and the behavior of its arithmetic operators.
Your programs are the same on every platform---there are no data type
incompatibilities across hardware and software architectures.

The architecture-neutral and portable language platform of Java technology is
known as the Java virtual machine. It's the specification of an abstract
machine for which Java programming language compilers can generate code.
Specific implementations of the Java virtual machine for specific hardware and
software platforms then provide the concrete realization of the virtual
machine. The Java virtual machine is based primarily on the POSIX interface
specification---an industry-standard definition of a portable system interface.
Implementing the Java virtual machine on new architectures is a relatively
straightforward task as long as the target platform meets basic requirements
such as support for multithreading.

\subsection High Performance.
Performance is always a consideration. The Java platform achieves superior
performance by adopting a scheme by which the interpreter can run at full speed
without needing to check the run-time environment. The automatic garbage
collector runs as a low-priority background thread, ensuring a high probability
that memory is available when required, leading to better performance.
Applications requiring large amounts of compute power can be designed such that
compute-intensive sections can be rewritten in native machine code as required
and interfaced with the Java platform. In general, users perceive that
interactive applications respond quickly even though they're interpreted.

\subsection Interpreted, Threaded, and Dynamic.
The Java interpreter can execute Java bytecodes directly on any machine to
which the interpreter and run-time system have been ported. In an interpreted
platform such as Java technology-based system, the link phase of a program is
simple, incremental, and lightweight. You benefit from much faster development
cycles---prototyping, experimentation, and rapid development are the normal
case, versus the traditional heavyweight compile, link, and test cycles.

Modern network-based applications, such as the \HotJava~Browser for the World
Wide Web, typically need to do several things at the same time. A user working
with HotJava Browser can run several animations concurrently while downloading
an image and scrolling the page. Java technology's multithreading capability
provides the means to build applications with many concurrent threads of
activity. Multithreading thus results in a high degree of interactivity for the
end user.

The Java platform supports multithreading at the language level with the
addition of sophisticated synchronization primitives: the language library
provides the Thread class, and the run-time system provides monitor and
condition lock primitives. At the library level, moreover, Java technology's
high-level system libraries have been written to be thread safe: the
functionality provided by the libraries is available without conflict to
multiple concurrent threads of execution.

While the Java Compiler is strict in its compile-time static checking, the
language and run-time system are dynamic in their linking stages. Classes are
linked only as needed. New code modules can be linked in on demand from a
variety of sources, even from sources across a network. In the case of the
HotJava Browser and similar applications, interactive executable code can be
loaded from anywhere, which enables transparent updating of applications. The
result is on-line services that constantly evolve; they can remain innovative
and fresh, draw more customers, and spur the growth of electronic commerce on
the Internet.

\section The Java Platform---a New Approach to Distributed Computing.
Taken individually, the characteristics discussed above can be found in a
variety of software development platforms. What's completely new is the manner
in which Java technology and its runtime environment have combined them to
produce a flexible and powerful programming system.

Developing your applications using the Java programming language results in
software that is portable across multiple machine architectures, operating
systems, and graphical user interfaces, secure, and high performance. With Java
technology, your job as a software developer is much easier---you focus your
full attention on the end goal of shipping innovative products on time, based
on the solid foundation of the Java platform. The better way to develop
software is here, now, brought to you by the Java platform.

\bye

